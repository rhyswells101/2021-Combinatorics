\documentclass[a4paper,12pt]{article}
\newcounter{example}[]
\newenvironment{example}[1][]{\refstepcounter{example}\par\medskip
 \newcounter{example}[]
\newenvironment{exercise}[1][]{\refstepcounter{exercise}\par\medskip
   \noindent \textbf{Exercise~\theexample. #1} \rmfamily}{\medskip}  \noindent \textbf{Exercise~\theexample. #1} \rmfamily}{\medskip}
   %%%%%%%%%%%%%%%%%%%%%%%%%%%%%%%%

%%%%%%%%%%%%%%%%%%%%%%%%%%%%%%%%%%
\usepackage[utf8]{inputenc}
\usepackage[english]{babel}
\usepackage{tikz-cd}
\usepackage{amsmath,amsfonts,amssymb,amsthm}
\usepackage{mathtools}
 \usepackage{float}
\usepackage{amsthm}
\usepackage{cite}
\usepackage{datetime} % British format dates
\usepackage[cm]{fullpage}
\usepackage{url}
\usepackage{hyperref}
\usepackage{stackrel,amssymb,amsmath}
\usepackage[nottoc]{tocbibind}
\usepackage{pgfplots}
\usepackage{rotating}
\usepackage[autostyle]{csquotes}
\usepackage{natbib}
\usepackage{graphicx}
\usepackage{natbib}
\usepackage{graphicx}

\newtheorem{problem}{Problem}
\newtheorem{attempt}{Attempt}


\newtheorem{theorem}{Theorem}[section]
\newtheorem{corollary}{Corollary}[theorem]
\newtheorem{lemma}[theorem]{Lemma}
\newtheorem{proposition}[theorem]{Proposition}
\theoremstyle{definition}
\newtheorem{definition}{Definition}[section]
\theoremstyle{indented}
\newtheorem*{remark}{Remark}
\newenvironment{titlemize}[1]{%
  \paragraph{#1}
  \begin{itemize}}
  {\end{itemize}}
  
  \usepackage[T1]{fontenc}
\usepackage{imakeidx}
\makeindex[columns=3, title=Alphabetical Index, intoc]
  
  
  %%%%%%%%%%5
\newcommand{\rightarrowdbl}{\rightarrow\mathrel{\mkern-14mu}\rightarrow}

\newcommand{\xrightarrowdbl}[2][]{%
  \xrightarrow[#1]{#2}\mathrel{\mkern-14mu}\rightarrow
}
%%%%%%%%%%%%%%5

\title{$\Tilde{P_n}$ and $P_n$}
\author{Rhys Wells}
\date{\today}

\begin{document}

\maketitle
\tableofcontents

\section{Jon's epsilon argument}

This is a test

Consider the map $\tilde{P_n} \xrightarrow{\alpha}\tilde{P_{n+1}}$ where $f \mapsto \tilde{f}$.In this section we consider maps $\epsilon$ taking $f \mapsto \tilde{f}$. Two special cases are $\epsilon=0$ with $\epsilon(S)=0$ for all $S$ and $\epsilon=1$  with $\epsilon(S)=1$ for all $S \in \mathcal{P}^{+}_{n}$. Note by $\tilde{P_n} \xrightarrow{\epsilon=0}\tilde{P_{n+1}}$ this is a surjection. There is a restriction $\tilde{P_{n+1}} \xrightarrow{res}\tilde{P_{n}}$ which forgets the values on $S$ which contain the element $n+1$. I have done by hand the $\tilde{P_{4}}$ to $\tilde{P_{3}}$ case, and $\tilde{P_{5}}$ to $\tilde{P_{4}}$. Extending from $f$ to $\tilde{f}$ one assigns by $\epsilon$ for each $A \in \mathcal{P}^{+}_{n}$

Only looking at positive values $f:\mathcal{P}_n^{+} \rightarrow \mathbb{N}\cup \{0\}$. $\epsilon(A)=0$ or $\epsilon(A)=1$ depending on the $A$. Set $\tilde{f}(A) = f(A) $  and $\tilde{f} ( A \cup \{n+1\}) = f(A) +\epsilon(A)$ for $A \in \mathcal{P}_n^{+}$

\begin{titlemize}{Questions}
\item For an extension $e \in \epsilon^{-1}_{f}(1)$ is $\tilde{f}$ MSA ? 
\end{titlemize}

\begin{example}
$\tilde{f{(\{1,4\}) = f(1)+\epsilon(\{1\})$.
\end{example}

Each epsilon depends on the structure of $f$ 

Describe the map $\tilde{P}_{n} \rightarrow \tilde{P}_{n+1}$.

To see that $\tilde{f}$ is MSA. Note $f(A)=\tilde{f}(A)$ for $n+1 \notin A$.
Recall $\tilde{f} (\{n+1\})=0$ and consider, $$f(A)= \tilde{f}(A) \le \tilde{f}(A \cup \{n+1\} \le \tilde{f}(A)+1=f(A)+1.$$ 

$\tilde{f}(A) \le \tilde{f}(A \squp {n+1}) \le \tilde{f}(A)+1$ as $\tilde{f} (\{n+1\})=0$, subtracting
$0 \le \tilde{f}(A \squp {n+1}) -\tilde{f}(A) - \tilde{f} (\{n+1\}) \le 1$. then use fact $f$ is MSA. 


every time we add a the new element ${n+1}$ to the subset, the new value is either increases by $1$ or $0$.

\begin{example}
Take $f \in \tilde{P_3}$ to be $f(12)=1, f(13)=0$ and $f(23)=0$ with $\epsilon(12)=0,\epsilon(13)=0,\epsilon(123)=0$ and $\epsilon(23)=1$
then $\tilde{f}$ will be 

$$f(12)=1, f(13)=0, f(23)=0, f(123)=1, f(124)=1, f(134)=0, f(234)=1, f(1234)=1.$$ 
\end{example}

Put structure of given $f$. We now put an ordering on a specific $f$ in $P_n$ for how subsets relate by their values.
this is connected to MSA def with $f(I\sqcup J)$ but taking $I \scup J$ to be $B$ and $A$ and $B \setminus A$ to be $I, J$. 
Example for $f \in \tilde{P_4}$ with $f(13)=1$ and $f(123)=1$ else $f(S)=0$. 
in this example $f$ is maximal w.r.t the subset $23 \subset 123$., and $f(123)=f(12)+f(3)$ equiv $1=0+1$  and similar for $3 \subset 123$.
and $f$ is minimal w.r.t $13 \subset 123$. as $f(123) = f(3)+f(12)+1 \equiv 1=0+0+1$
for $A,B \subset [n]$, if $A \subset B$ then 
$f$-minimal if $f(B)=f(A)+f(B \setminus A)$
$f$-maximal if $f(B)=f(A)+f(B \setminus A) +1$ 
        
What does the ordering on $P_n$ what to do with $P_{n+1}$? 


\begin{lemma}

Fix $f$. We claim if $A \subset B$ and $f$-minimal then $\epsilon(A)=1$ then $\epsilon(B)=1$
\end{lemma}
\begin{proof}

    Consider $\tilde{f}(B\cup\{n+1\})$,
    

    $$f(B)+\epsilon(B) = \tilde{f}(B\cup\{n+1\}) \ge \tilde{f}(A\cup\{n+1\}) + \tilde{f}(B \setminus A)$$ 
    
    $\tilde{f}(B \setminus A)$ has no $n+1$ element therefore $$\epsilon(B)\ge\epsilon(B\A))$.
    why this greater than ? $f$-min but do not know if $\tilde{f}$ is $f$ min 
    using using first less the $0$ in  $0 \le f(I\sqcupJ) -f(I)-f(J)\le 1$.
    we do not know if $\tilde{f}$ is MSA? yes it is. 
    note 
    $\tilde{f}(B \setminus A)$ 
    $\tilde{f}(A\cup\{n+1\}) = f(A)+\epsilon(A)+f(B\A) \overset{\mathrm{f-min}}{=} f(B)+\epsilon(A)$
    Therefore $f(B)+\epsilon(B) \ge f(B)+\epsilon(A)$
    therefore $\epsilon(B) \ge\epsilon(A)$
\end{proof}

Similarly if $A \subset B$ is $f$-max then$\epsilon(b)=1$ then$\epsilon(A)=1$ . 

Denote the preimage of elements $A \in \mathcal{P}_n^{+}$ where $\epsilon(A)=1$ to be  $\epsilon^{-1}(1)$. Consider the poset of inclusion of $ \mathcal{P_n^{+}$, and by definition of $\mathcal{P}_n^{+}$ we have
$$\epsilon(A)^{1} = \{ \text{upward closed under $f$-minimal and downward closed under $f$-maximal}\}$$
where upward and downward closed is relative to inclusion of poset of $ \mathcal{P_n^{+}$.

Therefore $\epsilon(A)^{1}$ satisfies for $A \subset B$, $\epsilon(A)=1 \implies\epsilon(B)=1$ and $\epsilon(B)=1 \implies\epsilon(A)=1$ respectively. We denote this in the poset as
green edges ($f$-min) and red edges ($f$-max).

\begin{example}
Fix $n=3$ and for $S=\{12\}$, $\{123\}$ set $f(S)=1$ and rest $0$. 
Consider the poset for $\mathcal{P}_3^{+}$ and define extensions to be $\epsilon^{-1}(1)$. The possible extensions for this $f$ are: 

\begin{align*}
\epsilon^{-1}(1) = \{& \\
&\emptyset,\\
&\{\{1\},\{13\}\},\\
&\{\{2\},\{23\}\},\\
&\{\{3\},\{13\},\{23\}\},\\
&\{\{1\},\{2\},\{13\},\{23\}\},\\
&\{\{1\},\{3\},\{13\},\{23\}\},\\
&\{\{2\},\{3\},\{13\},\{23\}\},\\
&\{\{1\},\{2\},\{3\},\{13\},\{23\}\},\\
&\{\{13\}\},\\
&\{\{23\}\},\\
&\{\{1\},\{2\},\{12\},\{13\},\{23\},\{123\}\},\\
&\{\{1\},\{2\},\{13\},\{13\},\{23\},\{123\}\},\\
&\{\{1\},\{2\},\{13\},\{12\},\{13\},\{23\},\{123\}\}
\end{align*}

There are $13$ extensions and this matches the $P4$ to $P3$ file. 
\end{example}

There is a symmetry in the $f$-min\ maximally conditions that for all $A \subset B$ if $A \rightarrow B$ is green ($f$-min) then so is $B\setminus A$ (ie has the same colour). This smell a lot like $\mathbb{Z}/2\mathbb{Z}$ action. 

Why are there multiple extensions ? 

why the need for upward/downward closed ?

There are losts of possible $e$'s and we want those such that $f$ is recoverable from $\tilde{f}$. There is a spectrum of $e$'s $0 \le e \le \underline{1}$ but not all are usefull for the MSA condition hence we need structure on $f$ and choose $\epsilon^{-1}_{f}(1)$ :=extensions for each $f$. In order to determine all extensions find all $\tilde{f}$ from a single node (core "basis" extensions) of green red diagram (where$\epsilon(A)=1)$ , then take all combination of these core extensions. 
Extensions we like are $A \in \mathcal{P}_n$ 
extension are from directed green red diagramand induce further$\epsilon(A')=1$ by $f$ min\maximality . 

Is there a link between $\epsilon^{-1}(1)$ and strictly nested condition of simplicacal MSA functions or $f$ - min  maxand strictly nes ted. 

There exists inherant structure on each $f$ on wether they grow or not which is given byy the previos extensions $\tilde{P_n} \xrightarrow{E} \tilde{P_{n+1}}$.
Where $E$ is a bundle of $e$ ( $E(f)=e_f) e_f$ is not well defined though.


\hline

Check stanley enumurative combinatorics pdf .

should be studying these epsilons. skyscraper sheaves? 
As we have the constant $\epsilon =0$ or $\epsilon=1$  and surjective so $|\tilde{P}_{n+1}| \ge 2|\tilde{P}_{n}|$.

\section{Burnside Lemma}

Investigate burnsides lemmas use:

$#orbits = |\tilde{P_n} / S_n| = \frac{1}{n} \sum_{\sigma \in S_n} |\tilde{P_n}^{\sigma}|$

for $\sigma$-fixed points.

\section{Group actions}

$\mathbb{Z}/2\mathbb{Z}$ 

$f \mapsto \bar{f}$
with $\bar{f}(A)=|A|-1-f(A)$. 

Example :

For $\tilde{P_3}$ there are $3!$ orbits 

$|\tilde{P_3}^{3-cycle}|=4, |\tilde{P_3}^{2-cycle}|=6, |\tilde{P_3}^{id}|=10$

and there are two $3$-cycles, three $2$-cycles. 

$\frac{1}{6}(2*4 + 3*6+ 10)=6$



\section{Avenues}

Consider abstraction of the problem to combinatoircal Species, with the bijection, 

$\tilde{p}:FinSet : FinSet $
 with $A \mapsto \tilde{p}_A$ which related to $\sum^{\infin}_{n=0} |\tilde{p_n}| x^n$. where we want coefficents. We build up species by generating functions. 


\subsection{$S_n$ action on $\tilde{P_n}$}

Consider the group action $S_n$ action on $\tilde{P_n}$$ for $f \in \tilde{P_n}$ with $f: \mathcal{P}_n^{+} \rightarrow \mathbb{Z}$. Where  $\sigma \cdot f(S) = f(\sigma(S))$.

\begin{titlemize}{Questions}

\item Is $\sigma(f) \in \tilde{P_n}$? is it MSA?. Show that $S_n$ action respects MSA condition.

\end{titlemize}

\subsection{$\mathbb{Z}_2$ action on $\tilde{P_n}$}

Consider the $\mathbb{Z}_2$ action on $\tilde{P_n}$. Elements of $\mathbb{Z}_2$ are $\{0,1\}$ and $1$ is the non trivial element. 

$$f \mapsto \tilde{f} $$

where, 

$$ \tilde{f}(A) = |A| -1-f(A)$$

i $$1 \cdot (f) (A) = |A| -1-f(A)$$


\begin{example}
Consider $f$ with $R_f=\emptyset$ associated to $\{1,3,5\}$ and $\{2,4,5\}$. By permuting the fixed term $5$. There are $15$ other such functions. Along with the $Z_2$ action we get  $30$ such functions. To check these are all see Linear programming code. Aiming for 1924 such functions.
\end{example}



\subsection {In $P_n$ the $S_n$ action on $\mathbb{R}^n$}

where $\sigma(x_1,\dots ,x_n) = (x_{\sigma(1)},\dots ,x_{\sigma(n)}) $ acting on the axis.

This induces and action on polytopes. One has for all $P \in P_n$, $\sigma(P) \in P_n$. To show this its enough to show $\sigma(H)$ is a hyperplane in $H(S,k)$ for all $H \in H(S,k)$. 

Recall $$H(S,k) = \left\{ H_{S,k} \right\}_{\mathcal{P}_{n}^{+}$$

where for a specific hyperplane $H_{S,k} = \{\underline{x} \quad | X_S =k\}$ for specific $S,k$. 

and $\sigma(H_{S,k}) = H_{\simga{S},k}$. 

\begin{remark}
Where on hyperplanes $X_S \mapsto X_{\simga(S)}$. In $P_n$ only considering singletons ie $x_{\{i\}} \mapsto x_{\sigma(\{\ i\})}$. In $\tilde{P_n}$ case there is richer poset structure on $\mathcal{P}^{+}_{n}$. 
\end{remark}

\subsection{$\mathbb{Z}_2$ action on $P_n$}

Consider the action $Z_2$ on $\mathbb{R}^n$ by reflection around the origin by $1\cdot (x_i)=-x_i$. One has $$1 \cdot H_{S_k} = H_{S,-k}.$$

\section{Simplicial $\tilde{P}_n$ }



\section{Nicola's consecutive terms}

Consecuative case is related to the cyclic ordering on necklace curves.How to get from general msa function to consecuative case? 

The number of extensions of $f:C_n \rightarrow \mathbb{Z}$ is constant and equal to $n$. For $C_1$ have $1$ extension and by inductions the total number is $n!$. 

The ordering $<_f$ incorporates jon's epsilon graph construction Show this.  ie which of jons are from the consecuative case. 

\begin{example}

Consider $f:C_3 \rightarrow \mathbb{Z}$ such that $f(\{1,2\})=1, f(\{2,3\})=0$ and $f(\{1,2,3\})=1$.
Reducing from epsilon graph for f.
Possible $\epsilon^{-1}(1) = \{\{1\},\{2,23\},\{1,2,3,23,123\},\{1,2,23\}, \{1,2,12,23,123\}, \{1,2,23,123\}\}$
of which there are $6$.

\end{example}

Let $C_n =\{  \{t,t+1,\dots , s\} |\quad \text{for all} \: 1\let\le s \le n\} $

Let $Q_n  = \{ f: C_n \rightarrow \mathbb{Z}| MSA s.t. f(\{J\})=0 for all J \in [n]\}$

$f$ being MSA means for all $A,B \in C_n$ such that $A \subset B$ and $B\setminus A \in C_n$

$$0 \le f(B) -f(A)- f(B \setminus A) \le 1$$

where 
we have f-min condition $0 = f(B) -f(A)- f(B \setminus A)$
we have f-max condition $1= f(B) -f(A)- f(B \setminus A)$

Claim: $|Q_n|= n!$

we show the claim by proving that each $f \in Q_{n-1}$ admits $n$ extensions $\tilde{f} \in Q_n$ (we say that $\tilde{f}$ extends $f$ if $\tilde{f}|_{Q_{n-1}} =f$. 

\begin{proof}
For $f \in C_{n-1}$, define the order relation $<_f$ on $[n-1]$ as follows:

$t <_f s$ if and only if either $t<s$ and $f(\{t,\dots n-1\}) = f(\{t,\dots s-1\}) + f(\{s,\dots n-1\})$ (f-min) or  $s<t$ and $f(\{s,\dots n-1\}) = f(\{s,\dots t-1\}) + f(\{t,\dots n-1\}) +1$ (f-max).


\begin{exercise} {exactly jons arguement for epsilon}
If $t <_f s $ show that if $\tilde{f}(\{s,\dots n\}) = f(\{ s,\dots , n-1\}) +1 $, then $\tilde{f}(\{t,\dots n\}) = f(\{ t,\dots , n-1\}) +1 $
(by msa)
\end{exercise}

Let $Q_n^{f} = \{\tilde{f} \in Q_n | \quad \tilde{f} \text{extends} f\}$

One has $Q_{n-1} \xrightarrow[]{\epsilon} Q_n^{f}  $

where $Q_n^{f} \subseteq Q_n$

By the exercise $\psi$ is a bijection. 

\begin{example}

\end{example}

\begin{example} {showing ordering}
Let $n=3$ with $f \in C_2$ with $f(\{1\})=0,f(\{2\})=0$ and $f(\{1,2\})=1$
Possible extensions $\tilde{f} \in C_3$

$$\tilde{f}_A \: \text{where} f(\{1,2\})=1,f(\{2,3\})=0,f(\{1,2,3\})=1$$
$$\tilde{f}_B \: \text{where} f(\{1,2\})=1,f(\{2,3\})=1,f(\{1,2,3\})=1$$
$$\tilde{f}_C \: \text{where} f(\{1,2\})=1,f(\{2,3\})=1,f(\{1,2,3\})=2$$

Consider $\{1,2\}$ 
Is $t=2 <_f 1=s$? 
take $s=1<2=t$ and $f(\{1,2\}) = f(\{1\}) + f(\{2\}) +1$
$1=0+0+1$ therefore $t=2 <_f 1=s$. 

$f(12)< \tilde{f}(123)$.

Consider if $\tilde{f}(\{1,2,3\}) = f(\{1,2\}) +1$ (which true for $\tilde{f}_C$ from exercise) then $\tilde{f}(2,3) = f(\{2\}) +1$ ($1=0+1$).

\end{example}

\begin{example}
Let $n=3$ with $f \in C_2$ with $f(\{1\})=0,f(\{2\})=0$ and $f(\{1,2\})=0$
Possible extensions $\tilde{f} \in C_3$

$$\tilde{f}_A \: \text{where} f(\{1,2\})=0,f(\{2,3\})=0,f(\{1,2,3\})=0$$
$$\tilde{f}_B \: \text{where} f(\{1,2\})=0,f(\{2,3\})=0,f(\{1,2,3\})=1$$
$$\tilde{f}_C \: \text{where} f(\{1,2\})=0,f(\{2,3\})=1,f(\{1,2,3\})=1$$

Is $t=1 <_f 2=s$? 
take $t=1<2=s$ and $f(\{1,2\}) = f(\{1\}) + f(\{2\}) $
$0=0+0$ therefore $t=1 <_f 2=s$. 

Given $t=1 <_f 2=s$

$\tilde{f}(23) = f(2)+1$
which is true for $\tilde{f}_C$ and $\tilde{f}(123)=f(12)+1$ which is true. 

$f(12)< \tilde{f}(123)$.

Consider if $\tilde{f}(\{1,2,3\}) = f(\{1,2\}) +1$ (which true for $\tilde{f}_C$ from exercise) then $\tilde{f}(2,3) = f(\{2\}) +1$ ($1=0+1$).

\end{example}

\begin{example}
Let $n=4$
$f(12)=0, f(23)=1,f(123)=1$
Then $3 <_f 1 <_f 2$ ,

For $3<_f 1$ we take $1<3$  and $f(123)=f(12)+f(3)+1$

And for $1<_f 2$ take $1<2$ and $f(123)=f(1)+f(23)$

These are those extensions of $f$ in $C_4$ extending by $4$.

$\tilde{f}_A (\epsilon =(0,0,0)),\tilde{f}_A (\epsilon =(1,0,0)) ,\tilde{f}_A (\epsilon =(1,0,1)) ,\tilde{f}_A (\epsilon =(1,1,1))$  

For each the condition inside max set of $\psi$ map.

$$\tilde{f}_A \: \text{where}\: f(\{3,4\})=0,f(\{2,3,4\})=1,f(\{1,2,3,4\})=1$$

The following do not hold $f(3)< \tilde{f}(34), f(23)< \tilde{f}(234), f(123)< \tilde{f}(1234)$

$$\tilde{f}_B \: \text{where}\: f(\{3,4\})=1,f(\{2,3,4\})=1,f(\{1,2,3,4\})=1$$
The following holds $0=f(3)< \tilde{f}(34)=1$
$$\tilde{f}_C \: \text{where}\: f(\{3,4\})=1,f(\{2,3,4\})=1,f(\{1,2,3,4\})=2$$
The following holds $0=f(3)< \tilde{f}(34)=1,f(123)< \tilde{f}(1234)$
$$\tilde{f}_D \: \text{where}\: f(\{3,4\})=1,f(\{2,3,4\})=2,f(\{1,2,3,4\})=2$$
The following holds $f(3)< \tilde{f}(34), f(23)< \tilde{f}(234), f(123)< \tilde{f}(1234)$. 



$\psi(\tilde{f}_A)=4$
$\psi(\tilde{f}_B)=3$
$\psi(\tilde{f}_C)=1$
$\psi(\tilde{f}_D)=2$
\end{example}

\begin{example}
Take $n=4$, $f(12)=1,f(23)=0,f(123)=1$

is $t=1<_f 2=s$? No 

take $1<2$ and $f(123)= f(1,2-1)+f(23)$ (not true)

Take $t=2<_f 1=s$ (True)
is $1<2$ and $f(123)=f(1)+f(23)+1$

Is $t=3<_f 1=s$, check $1<3$ and $f(123)=f(12)+f(3)$

So $2<_f 1 <_f 3$ wrt $f$  (this is some permutation of $[n]$ $p_f$ map.

\end{example}

\end{proof}


Is $\tilde{P_n}$ a glued together case of consecuative case? 
how do $f: C_{n-1} \rightarrow \mathbb{Z}$ glue to get $f \in \tilde{P_n}$. 

Consecutive terms are discussed in Geometry of fine g=1 jacobians paper. Add reference \cite[p.32]{pagani2020geometry}

This exercise shows $\psi$ is injective:

Set $p_f : [n] \rightarrow [n]$ such that,

$$p_f(1)<_f p_f(2) \cdots  <_f p_f(n)$$

\begin{titlemize}{Question}

\item Given an extension $\epsilon$, $f \mapsto \tilde{f}$ with $e({i})=1$ what happens to $p_f(i)$. 

Once we have growth by 1 at {i} this exercise shows all terms below relative to $<_f$ are also 1 (ie they cascade) (i in this case is the maximum the cascade point). Similar to jons e(A)=1 implies e(B)=1 fmin and fmax (z_2 action). 

\begin{example}
Consider
$$p_f(1)<_f p_f(2) \cdots  <_f p_f(n)$$
and 
$$1<_f \cdots 1<_f 0 <_f \cdots <_f 0 $$

The important part of information here is where $1<_f 0$ the rest is known by the form of the question. 

For $f \in \tilde{P}_n$ we would have a mix of terms (permutation by $S_n$) such as, 
$$1<_f 0 <_f 1 <_f  1<_f 0 <_f 1_f  <_f 0 \cdots  $$

If we tried to focus on $1<_f 0$ , cascade point we would lose information. ie consecuative case is nice. 

In consecuative case $C_n$ 

$\tilde{f}$ is unquiely determined by $f$ and the cascade point (where $1<_f 0$).

To show $  \psi : Q_n^{f} \rightarrow [n]$ a sujection, explicitly construct $\tilde{f}$. 

choose $\psi(\tilde{f}) = j$ and cascase wrt the jth position
$$1<_f \cdots 1<_f 0 <_f \cdots <_f 0 $$

Need to check $\tilde{f}$ is MSA use the fact that $\tilde{f}|_{Q_{n-1}} = f$ and that $\tilde{f}(A) =f(A)$ and $\tilde{f}(A \cup \{n\} )=f(A) + \epsilon(A)$. on restriction there is no $\{n+1\}$ term. 

How to relate $C_n$ to $\tilde{P_n}$.

denote case of consecuative functions by $C_n$. 

In $C_n$ each $f$ extends $n$ times (in $\tilde{P_n}$ each $f$ extends a nonconsant number of times).  

remember to check roam notes



\end{example}

\item Consider the e extensions on consecuative functions.
\item are consecuative functions MSA ? 

\end{titlemize}



\hline 

Studying $S_n $ action on $\bar{J}^{d}_{1,n}$ 

$$(1 ,\dots , 1,0 ,\dots0)$$ 

where cascade point occurs at the ith position. see \cite[Table on p.20]{pagani2020geometry}. We see material like,

$$(1 ,\dots , -1,0 ,\dots0)$$ 
and not 
$$(1 ,\dots , 1,0 ,\dots0)$$ 
Suscpect action of $\mathbb{Z}_2$. 


In the simplest case take $(1,0,\dots ,0)$ and after permutation $S_n$ get $(0,\dots ,1, \dots 0)$. And so on for more copies of 1. Is this space disjoint (stratification) 

\begin{exercise} {exactly jons arguement for epsilon}
If $t <_f s $ show that if $\tilde{f}(\{s,\dots n\}) = f(\{ s,\dots , n-1\}) +1 $, then $\tilde{f}(\{t,\dots n\}) = f(\{ t,\dots , n-1\}) +1 $
(by msa)
\end{exercise}

\begin{proof}
Note $f \in Q_n$ is MSA.

Assuming $t<_f s$ with $t<s $ and $f(\{t,\dots , n-1\}) = f(\{t,\dots , s-1\}) + f(\{s,\dots , n-1\})$

we have $\{t,\dots,s-1,s,\dots ,n-1,n\}$

As $\tilde{f}$ is MSA so equ1


$$ 0\le \tilde{f}(\{t,\dots , n\}) - \tilde{f}(\{t,\dots , s-1\}) - \tilde{f}(\{s,\dots , n\}) \le 1$$

From $f(\{t,\dots , n-1\}) = f(\{t,\dots , s-1\}) + f(\{s,\dots , n-1\})$ one gets $ 0= \tilde{f}(\{t,\dots , n\}) - \tilde{f}(\{t,\dots , s-1\}) - \tilde{f}(\{s,\dots , n\})$
Hence,
$\tilde{f}(\{t,\dots , n\}) = \tilde{f}(\{t,\dots , s-1\}) + \tilde{f}(\{s,\dots , n\}) $
as $\tilde{f}|_{Q_{n-1}} =f$ and inital hyposthiese $\tilde{f}(\{s,\dots n\}) = f(\{ s,\dots , n-1\}) +1 $

$$\tilde{f}(\{t,\dots , n\}) = f(\{t,\dots,s-1)+ f(\{s,\dots , n-1\}) +1+0$$
as $t<_f s$
$$ f(\{t,\dots,n-1\})+1$$


\begin{titlemize} {Question}

\item If we choose $1$ instead of $0$ in msa relation $\tilde{f}(\{t,\dots , n\}) = f(\{t,\dots,n-1\})+2$, leads to extra room wrt < in set $Mas(p: f(\{t,\dots,n-1\})<\tilde{f}(\{t,\dots,n\}) )$.  

\end{titlemize}

Case 2 (use $\mathbb{Z}_) $: Assume now $t<_f s$ with $s<t $ and $f(\{s,\dots , n-1\}) = f(\{s,\dots , t-1\}) + f(\{t,\dots , n-1\}) +1$

as $s<t$ we have $\{s,\dots,t-1,t,\dots ,n-1,n\}$

As $\tilde{f}$ is MSA so equ1

$$ 0\le \tilde{f}(\{s,\dots , n\}) - \tilde{f}(\{s,\dots , t-1\}) - \tilde{f}(\{t,\dots , n\}) \le 1$$

as  $f(\{s,\dots , n-1\}) = f(\{s,\dots , t-1\}) + f(\{t,\dots , n-1\}) +1$ as $\tilde{f}$ is an extension of $f$.
we focus on taking the 1 in this relation.

$\tilde{f}(\{s,\dots , n-1\}) - \tilde{f}(\{s,\dots , t-1\}) - \tilde{f}(\{t,\dots , n\}) =1$
equivelent to 
$\tilde{f}(\{t,\dots , n\}) = \tilde{f}(\{s,\dots , n\}) - \tilde{f}(\{s,\dots , t-1\}) - 1$
by the initial hypoptheses $\tilde{f}(\{s,\dots n\}) = f(\{ s,\dots , n-1\}) +1 $

called  below equation equ2
$\tilde{f}(\{t,\dots n\}) = f(\{s,\dots n-1\}) +1 -f(\{s,\dots t-1\}) -1$
rearranging 

$f(\{s,\dots , n-1\}) = f(\{s,\dots , t-1\}) + f(\{t,\dots , n-1\}) +1$
to
$f(\{s,\dots , t-1\})= f(\{s,\dots , n-1\}) - f(\{t,\dots , n-1\}) -1$
substitute this into equ2.  gives 


$\tilde{f}(\{t,\dots n\}) = f(\{s,\dots n-1\}) +1 -(f(\{s,\dots , n-1\}) - f(\{t,\dots , n-1\}) -1) -1$
$= f(\{t,\dots n-1\}) +1 $



\end{proof}

This exercise shows $  \psi : Q_n^{f} \rightarrow [n]$ is a bijection where (possible min see \cite{pagani2020geometry}) 
$$ \tilde{f} \mapsto Max\{ p : f(\{p,\dots, n-1\}) < \tilde{f}(\{p,\dots, n\})  \} $$
(maxwrt $<_f$)
or if $f(\{p,\dots, n-1\}) = \tilde{f}(\{p,\dots, n-1\}) \overset{(*)}{=} \tilde{f}(\{p,\dots, n\}) $ where $(*)$ have $\tilde{f}|_{Q_{n-1}} = f$,
$$ \tilde{f} mapsto \psi(f)=n$$  

Use the way nicola introduced the order here (see above)
Assume $\tilde{f}$ with ordering $<_f$ on $[n-1]$, with the ordering 
$$p_f(1)<_f p_f(2) \cdots  <_f p_f(n-1)$$

Assume $p_f(n-1) \in \{ p | \quad f(\{p,\dots ,n-1\}) < \tilde{f}(\{p,\dots ,n\}) \} $

by the exercise and the fact $p_f(n-2)<_f p_f(n-1)$ by the ordering we have imposed so, $f(\{p_f(i),\dots ,n-1\}) < \tilde{f}(\{p_f(i),\dots ,n\})$ 
Hence for $1\le i\le n-2$ we have $p_f(i) \in \{ p | \quad f(\{p,\dots ,n-1\}) < \tilde{f}(\{p,\dots ,n\}) \} $.
As $p_f(n-1)$ was take to be the max wrt $<_f$ and all terms in the set  $\{ p | \quad f(\{p,\dots ,n-1\}) < \tilde{f}(\{p,\dots ,n\}) \} $ so $\tilde{f} \mapsto p_f(n-1)$.

If we remove $p_f(n-1)$ we can inductively assign $\tilde{f} \mapsto p_f(i)$ and hence get $n-1$ distinct values.

AS soon as $$\tilde{f}(\{s,\dots,n\}) = f(\{s,\dots, n-1\} +1$$
then all $t$ such that $t <_f s$ also satisfy $\tilde{f}(\{s,\dots,n\}) = f(\{s,\dots, n-1\} +1$ and so are in $\{ p | \quad f(\{p,\dots ,n-1\}) < \tilde{f}(\{p,\dots ,n\}) \}$. Taking the max wrt $<_f$ returns $s$. 

\begin{titlemize}
\item For each $s \in [n-1]$ does $\tilde{f}(\{s,\dots,n\}) = f(\{s,\dots, n-1\} +1$ hold? i.e can only always find an extension $\epsilon(\{s,\dots n-1\})=1$ and $\epsilon(A)=0$ else?
\end{titlemize}



To show $|Q_n| = n!$ assume $|Q_{n-1}| = n-1!$. For each $f \in Q_{n-1}$ assoicate $n$ extensions $\tilde{f} \in Q_{n}^{f}$.


\subsetion{Mapping $\tilde{P_{n+1}}$ to $\tilde{P_n}$} 


Consider $\pi:\tilde{P_{n+1}} \rightarrow \tilde{P_n}$ with $f \mapsto \pi(f)$ with $\pi(f)(S)=f(S)$ for $S \in \mathcal{P}_n^{+}$. 

This gives a way of decomposing $\tilde{P_{n+1}}$ into extensions on $f \in \tilde{P}_n$. Where 
$$\tilde{P_{n+1}} = \bigsqcup_{f \in \tilde{P_{n}}} \pi^{-1}(f_i). $$
 
 
 The $S_n$ orbits of $\tilde{P_n}$ are finer than characteristic orbits $S_n \cdot f \subset Char(f)$. In $\tilde{P_4}$ the number of $S_n$ orbits is $26$.
 
 Use burnsides lemma to count the number of orbits for S_4.

\printindex

\bibliographystyle{alpha}
\bibliography{bibtex}



\end{document}



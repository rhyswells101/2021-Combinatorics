\documentclass[a3paper,12pt]{article}
\usepackage[utf8]{inputenc}
\usepackage[english]{babel}
\usepackage{tikz-cd}
\usepackage{amsmath,amsfonts,amssymb,amsthm}
\usepackage{mathtools}
 \usepackage{float}
\usepackage{amsthm}
\usepackage{cite}
\usepackage{datetime} % British format dates
\usepackage[cm]{fullpage}
\usepackage{url}
\usepackage{hyperref}
\usepackage{stackrel,amssymb,amsmath}
\usepackage[nottoc]{tocbibind}
\usepackage{rotating}
\usepackage[autostyle]{csquotes}
\usepackage{natbib}
\usepackage{graphicx}
%\usepackage{natbib}
\usepackage{graphicx}

\newtheorem{problem}{Problem}
\newtheorem{attempt}{Attempt}
\newtheorem{theorem}{Theorem}[section]
\newtheorem{corollary}{Corollary}[theorem]
\newtheorem{lemma}[theorem]{Lemma}
\newtheorem{proposition}[theorem]{Proposition}
\theoremstyle{definition}
\newtheorem{definition}{Definition}[section]
%\theoremstyle{indented}
\newtheorem*{remark}{Remark}
\newenvironment{titlemize}[1]{%
  \paragraph{#1}
  \begin{itemize}}
  {\end{itemize}}
  
\theoremstyle{definition}
\newtheorem{example}{Example}[section]  
\theoremstyle{definition}
\newtheorem{exercise}{Exercise}[section]  

\title{Combinatorics of $\tilde{P}_n$}
\author{Rhys Wells}
\date{\today}

\begin{document}

\maketitle
\tableofcontents

\subsection{returning 300821}

Since we want to classify the genus 2 case (i.e aim to understand FCJ over $\overline{M}_{2,1}$). Therefore we will need to do checks (which are: can we always find a phi for sigma(G) for all assingments (around origin is enough)?) for each graph in the stratification.

We already know the vine curve case. 

\subsection{Meeting 230821}

We only care about when sigma(G) is of size of complexity the oda result for non-degernate get the complexity (only care about non-degenerate otherwise not fine). 

In the triangle case when See example.pdf for a case where the d'-tuple are all different but give sigma(G) of size 3. So far I have only looked at the d'tuple kwith two different starting d' conditions. We want to consider the case where we can have five different such assignments in the d'-tuple.

If we get sigma(G) =5 woudl be good to show picture of sig'(di,gammai) decomposiniosn and overlap.

\subsection{Tropical Jacobian}

See \cite{oda1979compactifications} for a description.

$$C_1(G: \mathbb{R}) \righarrow H_1(G:\mathbb{R})$$

$$Jac(G)=H_1(G:\mathbb{R})/ H_1(G:\mathbb{Z})$$



\subsection{200821 returning from onenotes and coding}

Fix an origin in $S^{-2}(G)$ this is $d2'=(0,-2,0)$.

assign a ordering to spanning trees start with $Gamma_1=12,23$ then $Gamma_2=12,13$, $Gamma_3 =12,13$, $Gamma_4 =23,13$,$Gamma_5 =23,13$.

Want to find a $d-tuple$ ie (d2',d1',d1',d1',d1') or (d1',d2',d1',d1',d1') so that $\sigma(G)$ has 5 elements. Therefore we just need d1'. Focus on (d2',d1',d1',d1',d1') first

Set d1' to be arbitrary of the form $d1'=(a,-a-b-2,b)$

Next get $\sigma'(d2',\Gamma_1)$ with 3 terms and $\sigma'(d1',\Gamma_2)$ with 4 terms, and consider where overlaps occur.Suppress the spanning tree. In particular we want to exclude all but finitely many $d \in S^{-2}(G)$ (\cite{klivans2018mathematics}?). To do so we look at the overlaps of $\sigma'(d2')$ $\sigma'(d1')$ (remember there may be another terms added to $\sigma(G)$ by union with $\sigma'(d1',\Gamma_4)$).

\begin{example}
For a large, ie a=1000 then 
$$$|\sigma'(d2')$ \cap $\sigma'(d1')$|=7.$$
and there are no common terms.
\end{example}

In order to get $|\sigma(G)|=5$ we need atleast 2 overlaps of $\sigma'(d2',\Gamma_1)$ $\sigma'(d1',Gamma_2)$. Therefore we wish to exclude those with $|\sigma'(d2') \cap \sigma'(d1')|\le 1$.

To see whether there are overlaps, we can go by inspection starting with low a terms.

For any a term we see we never get 3 overlaps.

if $|\sigma'(d2') \not \subseteq \sigma'(d1')|$ then $|\sigma'(d2') \cup \sigma'(d1')|\ge 5$.

\begin{example}
In this case a sharp bound exists for a and b to be 
$a=\{0,1\}$ and $b=\{-1,0,1\}$ (these are independant).

Instead of checking overlaps explicitly we can use the metric method (algo) to look for the smallest a,b terms where no overlap occurs (then exlude these).

We cant take the metric as d1' are not defined yet as it is a infinte set and so the k the distance can be infinite. 

\end{example}

\begin{titlemize}{What I have done:}
\item For theses $\sigma(G)$ for -3<a,b<3 see if we can find a $\phi$ (not necessarily the average) that generates $\sigma(G)$. I.e For d in sigma(G) check $|d_1 -\phi_1|<3/2,|d_2 -\phi_2|<2,|d_3 -\phi_3|<3/2$ and $deg(\phi)=0$. 
\item Done 1): For -3<a,b<3 I checked if $|\sigma(G)|=5$. There are none.
\item Re-did the overlap argument in 1) for (d1,d2,d1,d1,d1) did not find $|\sigma(G)|=5$.
\end{titlemize}

\begin{titlemize}{To do:}
\item For theses $\sigma(G)$ for -3<a,b<3 see if we can find a $\phi$ (not necessarily the average) that generates $\sigma(G)$. I.e For d in sigma(G) check $|d_1 -\phi_1|<3/2,|d_2 -\phi_2|<2,|d_3 -\phi_3|<3/2$ and $deg(\phi)=0$. 
\item Ans: For d in $\sigma(G)$ plot components of all d's and try to find a phi compnents by perturbing them. What happens is that you get two d_i(v) and d_j(v) so that if you move \phi(v) the then either d_i(v) or d_j(v) no longer satisfy the phi in-equalities.
\end{titlemize}

d1' terms near d2' which are semistable wrt taking phi the average are

$$\{(0,-3,1),(0,-1,-1)(1,-1,-2),(1,-2,-1),(1,-3,0),(1,-4,1),(0,-2,0)\}$$


\begin{remark}
For a general graph we can use the d infinity metric

$$max \{k \in \mathbb{Z} \mid |d_i(v) - d_j(v) | <k , \forall v \in V\}$$
\end{remark}

\subsection{Meeting Notes 170821}
Want to focus on why we are caring about the overlaps.

look at the intersection of 

For G1 we have two assignments (d2,d1,d1,d1,d1) an
d (d1,d2,d1,d1,d1) where $\Gamma_1=G((v1,v2),(v2,v3))$.

To make the problem finite nicola suggests looking for d1 terms locally around origin (-2,0,0) as those 

Let D=3

\begin{titlemize}{Tasks}
\item 
\end{titlemize}

\subsection{metrics}
% To get finite degree -g d terms to test.

For general graphs suggest using the maximum metric 
$$d((x1 , y1), (x2 , y2)) = max(|x1 - x2|, |y1 - y2| ).$$
to make the problem finite.

For d1 in $S^{-2}(G)$ take arbitrary like d1=(a,-a-b-2,b)
get $|a|<k$, $|b|<k$ and $|a+b|<k$.

not for d-terms in d-tuple but for elements in $\sigma^{'}(\Gamma_i,d_i)$ $\sigma^{'}(\Gamma_j,d_j)$


\subsection{Meeting Notes 020821}

\subsubsection{Tasks}

\begin{titlemize}{Find examples to test if $\psi_G$ is surjective.}
\item Using the current graph example which I want to study whether $\psi_G$ is surjective is discussed in \cite[(4),p42]{oda1979compactifications}.
\item For $G$ we have $Aut(G)=Z_2 \times Z_2$. Using the $Aut(G)$ we have two orbits of spanning trees $\Gamma_1$ and the rest. Let $\Gamma_1$ be the spanning tree with edges $v_2 - v_1$ and $v_3-v_2$. Pick a $d_2 \in S^{-2}(G)$ i.e. the origin=$(-2,0,0)$ (allowed up to translation).
\end{titlemize}

\begin{titlemize}{Coding ideas}
\item Reduce the problem to something finite that the computer can better handle. Consider the action of automorphism group of spanning trees. 
\item Code suggestion: try $(d_2,d_1,d_1,d_1,d_1)$ and $(d_1,d_2,d_1,d_1,d_1)$ with spanning trees in that given order. As in $S^{-2}(G)$ need to make a choice for the origin (unlike in S(G) where (0,0,0) can be taken). 
\item Results only give the case where ABKS gives $|\sigma(G)|=5.$.
\item Need to be able to decide when a graph is exhausted and when to move on. 

\begin{definition}
Let $\phi \in V(G)$, we say $d \in S(G)$ is $\phi$-(semi)stable if

\begin{equation*}\label{stabinequal}
\Bigl| \sum_{v \in Vert(G_0)} d(v) - \phi(v)\Bigr| \underset{(\le)}{<}  \frac{\# (E(G\setminus G_0) \cap E(G\setminus G_0^{c}) )}{2},
\end{equation*}

for all $\emptyset \subsetneq G_0 \subsetneq G $, where $G_0$ is a complete subgraph and $G_0^{c}$ denotes the complete subgraph on the complement vertices to $G_0$. Denote the set of $\phi$-stable degrees by
$$\sigma_{\phi}(G):=\{d : \text{Vert}(G) \rightarrow \mathbb{Z} \mid \phi \text{-stable}\} \subseteq S(G).$$
\end{definition}

\item Assume for $d_1 \neq d_2 \in S^{-2}(G)$ and let

$$D:=Max \{|d_1(v)-d_2(V)|>k, \forall v \in V\}$$.

% using < would make sure are closer together. But if choose > then we take all integers below k i.e k-1,k-2... but not 0 otherwise just ABKS.

Use this condition to get a finite list of $d_1 \in S^{-2}(G)$ terms to check.

And check that we $|\sigma(G)|=5$ and then check to see if axiom 1 holds.Note far away d terms are likely to be disjoint i.e not much overlap in $\sigma(G)$. See whiteboard for calcs. Try $D \in \{3,4,5\}$ (possibley 5 as its the complexity).

\end{titlemize}
\subsubsection{Notes}

\begin{titlemize}{Show $\psi_G$ is not surjective.}
\item We hope $\psi_G$ is not surjective otherwise theory is boring.
We only have a small pool of examples so far for testing if $\psi_G$ is surjective. 
\item Two possibilities: 
1) is surjective, then the theory is described completely by Oda.
2) not surjective, then we should take a thickening of the stability space V(G)  to recover $S \in \Sigma_G$ and $V(G)$ will be a shadow of this space.
\end{titlemize}



\subsection{Meeting Notes 270721}


\subsubsection{Tasks}

Construct $\tilde{\sigma}(G)$ with $d_i^{'}$ assignments not all the same (unlike in \cite{AN_2014}, if they are all the same but not 0 then translate). We wish to give a counterexample of lemma \ref{conject}. Break axiom 1 into two,

\begin{enumerate}
\item  there is at least one $e_i$ for each orbit,
\item no more that one $e_i$ for all orbits. (To show the counterexample Nicola was pushing me towards showing this)
\end{enumerate}



\begin{titlemize}{Question Harder}
\item Assuming the addition property $|\tilde{\sigma}(G)|=k(G)$ ("global property"). Consider $d^{'}_1$ and $d^{'}_2$ under these conditions do these satisfy axiom1?  (Try to disprove with basic examples by hand). I.e. find a counterexample for lemma \ref{conject}
\end{titlemize}

\begin{titlemize}{Ideas:}
\item  By taking $e_i,e_j \in \tilde{\sigma}(G)$ and showing they are both in the same orbit. When checking axiom 1 by hand Nicola recommends the greedy algorithm (in the case of four edges).
\item Taking completion of a given $d^{'}$. By fixing $d^{'}$ and constructing $\cup_{i \in I}\sigma^{'}(G)(d^{'},\Gamma_{i})$. Do this for $d^{'}_1$ and $d^{'}_2$ try example what do we get. 
\item (two cases $\Gamma_i=\Gamma_j$ or $\Gamma_i \neq \Gamma_j$)
\end{titlemize}

\begin{titlemize}{Checks}
\item How do we check if all $d_i^{'}$ are all the same and not just a translate of \cite{AN_2014}? Given a $\tilde{\sigma}(G)$ how do we know if this can be obtained in a trivial way (by a unique assignment $d^{'}$)? Given $e \in \tilde{\sigma}(G)$ take min on vertices to get $d^{'}$ and check if total degree is correct ($-g$).
\end{titlemize}

\begin{example}
In Example.pdf is an example with non-unique $d^{'}$. 
\end{example}

\begin{example}Here is an example where it works.
Consider the triangle with $$\tilde{\sigma}(G)=\{(1,0,0),(0,1,0),(0,0,1)\}$$
(This is in total degree 1 however will need to change) take then $(0,0,0)$.
\end{example}

\begin{titlemize}{Question Easier: Done}
\item Take $\tilde{\sigma}(G)$ not necessarily with $|\tilde{\sigma}(G)|=k(G)$ and show this fails axiom 1. See paper, using greedy see $E_{23}$ is winnable.
\end{titlemize}



% \subsubsection{Tertiary information}
% Interpreting the attached lengths of metric graphs. Metric graphs are given when considering "smoothenings" of nodal curves (the length is given by the $k$ integer of $xy=t^k$ reduction).

% \subsubsection{Notation}
% Call the starting terms $d^{'}$ assignments. 
% Integral divisors mean divisors are supported on the vertices.


\subsection{Generalised stability conditions}

We are now ready to define the conditions for generalised stability conditions $\sigma(G) \subseteq S(G)$. Let $\mathcal{ST}(G)$ denote the set of spanning trees of $G$. For a fixed spanning tree $\Gamma$, denote the set of source maps of $G \setminus \Gamma$ to be $\mathcal{O}_{\Gamma}=\{s:Edges(G \setminus \Gamma) \rightarrow Vert(G)$\} assigning a source vertex to each edge, equivalently let $\mathcal{O}_{\Gamma}$ denoted the set of orientations on $Edges(G \setminus \Gamma)$. In order to define $\sigma(G) \subseteq S(G)$ define for each spanning tree $\Gamma_i$ and $d^{'} \in S^{-g}(G)$ the following set of divisors

$$\sigma^{'}(\Gamma,d^{'}
):= \{ d \in S(G) \mid d= d^{'} + \sum_{l \in E(G\setminus \Gamma)} \delta _{s(l)} \text{ for } s \in \mathcal{O}_{\Gamma}  \}.$$

\begin{definition}\label{axioms12}
A generalised stability condition for $G$ is a subset $\sigma(G) \subseteq S(G)$ such that,

\begin{enumerate}
    \item the set $\sigma(G)$ is a complete set of representatives for the chip-firing action and
    
\item for each spanning tree $\Gamma_i$ for $i \in I$, there exists $d_i^{'} \in S^{-g}(G)$ such that 

$$\sigma(G) = \bigcup_{i\in I} \sigma^{'}(\Gamma_i,d_{i}^{'}).$$

\end{enumerate}
\end{definition}

\begin{titlemize}{Pointers}
\item In \cite[pg.6]{shen2017break} there is an outline for break divisors and potentially a better way of stating $\sigma(G)$ (metric graphs).
\item Break divisors are discussed in \cite[Definition 1.1.16]{KarlBreak} and correspond to $v$-rooted orientations \cite[Prop 1.2.20]{KarlBreak}, in \cite[Section 4.1.1]{KarlBreak} he discusses the poset of break divisors.
\item \cite[Chapter 3.1.2]{yuenMatriodsgeometric} discusses break divisors along the same lines we have in particular \cite[Prop 3.1.2]{yuenMatriodsgeometric}.
\end{titlemize}

\begin{titlemize}{Question}
\item Do our $\sigma(G)$ (shifted in degree) correspond to generalised break divisors as in \cite[Chapter 3.5]{yuenMatriodsgeometric}.
\end{titlemize}

\begin{definition} Denote the set of all generalised stability conditions by
$$\Sigma_{G} = \left\{ \sigma(G) \subseteq S(G) \mid \sigma(G) \text{ is a generalised stability condition} \right\}\subseteq 2^{S(G)}.$$
\end{definition}

We use \cite[Lemma 3.20]{Kass_2017} to give the well-defined map $\psi_G: Q_G \rightarrow \Sigma_G$. Here is a simplified version of it.

\begin{lemma}{\cite[Lemma 3.20]{Kass_2017}}
Let $\phi_1,\phi_2 \in V(G)$ with non-degenerate $\phi_1$. Then for all $d \in \sigma_{\phi}(G)$ is in $\sigma_{\phi_2}(G)$ if and only if $\phi_2 \in P(\phi_1)$. In particular the following map is well defined,
\begin{align*}
    \psi_G: Q_G &\rightarrow \Sigma_G\\
    \phi \in P &\mapsto \sigma_{\phi}(G).
\end{align*}
\end{lemma}

\begin{titlemize}{Side question}
\item Is $\psi_G$ injective by \cite[Lemma 3.20]{Kass_2017}?
% \item Do $\sigma_{\phi}(G)$ satisfy the axioms 1 and 2 of definition \ref{axioms12}, i.e. why are $\sigma_{\phi}(G) \in \Sigma_G$? What Nicola said:"This is geometrically motivated in \cite{oda1979compactifications}, in a proof of compactification (proper) where axiom 1 corresponds to universally closed and axiom 2 is separation."
\end{titlemize}

\begin{titlemize}{Main objective}
\item Show for each graph $G$, that all fine generalised stability $\sigma(G)$ of the form $\sigma_{\phi}(G)$ for some non degenerate $\phi \in V(G)$. Equivalently, that $\psi_G: Q_G \to \Sigma_G$ surjective for all $G$.
\end{titlemize}

\subsection{Constructing $\sigma(G)$}

Instead of answering the surjectivity question, we instead "try to construct $\sigma(G)$ from $\mathcal{ST}(G)$" to satisfy definition \ref{axioms12}. To construct a $\sigma(G)$ we use axiom 2 for a choice $d_i \in S^{-g}(G)$ for each given $\Gamma_i$, denote this constructed set as $\tilde{\sigma}(G)$. Let $k(G)$ be the complexity of $G$, to satisfy axiom 1 a necessary condition is $|\tilde{\sigma}(G)|=k(G)$. 

\begin{theorem}[Fact]
For any $G$, for each $\Gamma_i \in \mathcal{ST}(G)$ fix $d_i \in S^{-g}(G)$. If we construct the set $\tilde{\sigma}(G)=\{e_1,\dots,e_t\}$ from $\{d_1,\dots,d_{k(G)} \}$ by applying axiom 2, then 

$$t \ge k(G).$$
\end{theorem}

\begin{proof}
% These are my ideas:\\ 

% Take all $d_i$ to be the same, by translating take these to be 0. Then for each $i$ we have $e_i=\sum_{l \in E(G\setminus \Gamma_i)} \delta _{s(l)} \text{ for } s \in S_{\Gamma_i}$, as there are $k(G)$ spanning trees so $t \ge k(G)$\\

% "For each $d_i$ and $\Gamma_i$ there is at least one $\sum_{l \in E(G\setminus \Gamma_i)} \delta _{s(l)} \text{ for } s \in S_{\Gamma_i}$ and so $|\sigma^{'}(\Gamma,d^{'}
% ) |\ge 1$ (q: what stops them from all overlapping when we take the union? to form $\tilde{\sigma}(G)$)."(possible remove)\\

If we assume $\sigma(G)$ is a generalised break divisor, then this follows from \cite[Theorem 3.5.1]{yuen2017geometric} (I think $\mathcal{BD}_{I_G}(G)$ corresponds to $\tilde{\sigma}(G)$ for some $I_G$). Nicola has a proof of this.
\end{proof}

% \begin{example}
% To get $t=k(G)$ in $\tilde{\sigma}(G)$ we need to maximise overlaps when taking the union of $\sigma^{'}(\Gamma_i,d_i^{'})$. Finish the code to construct $\tilde{\sigma}(G)$ to give examples other than the triangle graph case.
% \end{example}

\begin{lemma}\label{conject}
Given a $G$ and $\tilde{\sigma}(G)$ such that $|\tilde{\sigma}(G)|=k(G)$, then it satisfies axiom $1$.
\end{lemma}

Note for any $G$ we have the set of divisors $\sum_{l \in E(G\setminus \Gamma)} \delta _{s(l)} \text{ for } s \in \mathcal{O}_{\Gamma}  $ for a given orientation, we will call the set of such divisors on all $\Gamma_i$, IBD. A special case is given in \cite[Thm 1.3, thm 4.21]{AN_2014} where $\sigma(G)=IBD$ is shown to satisfy lemma \ref{conject}, there all $d_{i}^{'}=0$ (which are not in $S^{-g}(G)$, so need to shift the degree in the definition \ref{axioms12}).\\

Suppose we have $\tilde{\sigma}(G)$ with $|\tilde{\sigma}(G)|=k(G)$ with $e_i,e_j \in \tilde{\sigma}(G)$. How do we know if these divisors are in separate orbits i.e form a complete set of representatives? In particular we wish to know if $E:=e_i-e_j \nsim 0$ by chip-firing equivalence (where $E \in S(G)$). We need to do this pairwise for all $e_i,e_j \in \tilde{\sigma}(G)$ to show they are different, then by the pigeon hole principle we have one for each class. \\

% (if $e_j \in |e_i|$ (a complete linear system $|D|=\{E \in Div(G)\mid E \le 0, E \sim D\}$)) in which case $\tilde{\sigma}(G)$ would not be a complete set of representatives.\\


\begin{definition}{\cite[Def. 1.14]{corry2018divisors}}
If D is winnable if and only if $D$ is linearly equivalent to an effective divisor (positive on each vertex). In total degree 0 the only D with all $D(v_i)\ge 0$ is $0$. 
\end{definition}


% \begin{remark}
% The Cori-Le Borgn algorithm (similar to Dhar) gives a bijection from $q$-reduced divisors to spanning trees.
% \end{remark}

% \begin{titlemize}{Greedy algorithm \cite[Section 3.1]{corry2018divisors}}
% \item Calculations take exponential time and is less efficient as only looks at a single vertex at a time.
% \item Doesn't give q-reduced only winnable (effective).
% \end{titlemize}

% \begin{titlemize}{Dhar's algorithm }
% \item Polynomial time.
% \item There exists a unique $q$-reduced divisor 
% \item More global in operation.
% \item Used in \cite{AN_2014}
% \end{titlemize}

Dhar's algorithm \cite[Section 3.4.1]{corry2018divisors} can be used to check if $E \sim 0$. For a divisor $E$, \cite[Theorem 3.7]{corry2018divisors} states we can obtain a unique $q$-reduced divisor and in particular we have the statement \cite[Theorem 3.8]{corry2018divisors}. Suppose $E$ is $q$-reduced (on all other vertices $E(v_i)\ge 0$) if $E(q)\ge 0$ then as $Deg(E)=0$ so $E=0$ (winnable). Therefore we aim check if $E(q)<0$ ($E$ is unwinnable).\\

\begin{remark}
By \cite[Corrolarly 4.9, (3)]{corry2018divisors} we have a bijection with acyclic orientations and maximal superstable configurations (which coincides with a $q$-reduced divisor where $\tilde{V}=V\setminus{q}$).
\end{remark}

\begin{titlemize}{Question}
\item Is there another way using the form of $e_i \in \tilde{\sigma}(G)$ and forming a bijection with $J(G)$ to show axiom $1$?
\end{titlemize}

\begin{titlemize}{Studying \cite{AN_2014} }
\item Investigate how \cite{AN_2014} show IBD are a section and use \cite[Theorem 1.3, Thm 4.21]{AN_2014} as a guide to show lemma \ref{conject}. Do they do something similar to $E(q)<0$? 
\item For $\sigma(G) \subseteq S(G)$, do we have the bijection $\sigma(G) \rightarrow Jac(G)$?
\end{titlemize}

% From the proof of \cite[Thm 4.21]{AN_2014} need to make use of metric graphs.

\subsection{Using Riemann Roch and $r(E)$}

Recall the definition of combinatorial rank in \cite[Definition 4.2]{baker2016degeneration}. By \cite[Exercise 5.1]{corry2018divisors} for $E\in S(G)$ we have $r(E) \le max\{-1,deg(E)\}=0$, so $r(E) \le 0$. If $r(E)=0$ then $E$ is barely winnable. By \cite[Exercise 5.2]{corry2018divisors} $E$ is principal and $e_i$ and $e_j$ are linearly equivalent \cite[p.20]{corry2018divisors}. Else $r(E)=-1$ and $E$ is unwinnable. Hence it is enough to show $r(E)=-1$.

\subsection{Overview of larger context}

\begin{enumerate}
    \item Compactified Jacobians are discussed in \cite[Section 2.2]{alexeev2004compactified} discussing the Oda paper.
    \item \cite[Section 3.3]{caporaso2019combinatorics} discusses the combinatorics of compactified Jacobians
\end{enumerate}

\begin{titlemize}{Universal case}
\item In the universal case have $V_{1,n} \cong \mathbb{R}^{n-1}$ over $\overline{M}_{1,n}$ and modulo translations we have $P_{1,n-1} \cong \mathbb{R}^n$ (careful with $\mathbb{R}^n$).
\item In the universal case we do have a counterexamples hence: 
$P_n$ thought of as polytope in vector space. Similarly can think of $\tilde{P}_n$ as a polytope in (possibley a vector space) (that contain $\mathbb{R}^n$) (Pic of X is T-bundle?).
\item $P_n \subseteq \tilde{P}_n$
\end{titlemize}

\begin{titlemize}{Single curve}
\item In normal case have ($V(G)$, $\sigma_{\phi}(G)$ and $P_n$.
\item Generalised case: For a single curve we have ($\Sigma(G)$) and in universal have ($\tilde{P}_n)$. 
\item Denote by $\Sigma_{\phi}$ the set of $\sigma_{\phi}(G)$.
\item For single curve case dont have counterexample, if had can propose thickening idea. 
\end{titlemize}

\begin{titlemize}{How to show $\sigma_{\phi}(G)\in \Sigma_G$.}
\item Ask about pages of Oda to read, focusing on compactifaction (open and properness correspond to axiom 2 and 1 for $\sigma_{\phi}(G) \in \Sigma_G$). Note in this paper $\phi$ are taken arbitrarily (can be on hyperplanes and double/triple.. intersections of hyperplanes). 
\end{titlemize}


\begin{titlemize}{Question}
\item In \cite{caporaso2019combinatorics} does $\overline{P}^{g}_{X}$ coincide with our notion of $\overline{J}(X)$?
\end{titlemize}

\printindex

\bibliographystyle{alpha}
\bibliography{bibtex}

\end{document}

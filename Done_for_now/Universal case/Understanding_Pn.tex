\documentclass[a3paper,12pt]{article}
\usepackage[utf8]{inputenc}
\usepackage[english]{babel}
\usepackage{tikz-cd}
\usepackage{amsmath,amsfonts,amssymb,amsthm}
\usepackage{mathtools}
 \usepackage{float}
\usepackage{amsthm}
\usepackage{cite}
\usepackage{datetime} % British format dates
\usepackage[cm]{fullpage}
\usepackage{url}
\usepackage{hyperref}
\usepackage{stackrel,amssymb,amsmath}
\usepackage[nottoc]{tocbibind}
\usepackage{rotating}
\usepackage[autostyle]{csquotes}
\usepackage{natbib}
\usepackage{graphicx}
%\usepackage{natbib}
\usepackage{graphicx}

\newtheorem{problem}{Problem}
\newtheorem{attempt}{Attempt}
\newtheorem{theorem}{Theorem}[section]
\newtheorem{corollary}{Corollary}[theorem]
\newtheorem{lemma}[theorem]{Lemma}
\newtheorem{proposition}[theorem]{Proposition}
\theoremstyle{definition}
\newtheorem{definition}{Definition}[section]
%\theoremstyle{indented}
\newtheorem*{remark}{Remark}
\newenvironment{titlemize}[1]{%
  \paragraph{#1}
  \begin{itemize}}
  {\end{itemize}}
  
\theoremstyle{definition}
\newtheorem{example}{Example}[section]  
\theoremstyle{definition}
\newtheorem{exercise}{Exercise}[section]  



\title{Combinatorics of $\tilde{P}_n$}
\author{Rhys Wells}
\date{\today}

\begin{document}

\maketitle
\tableofcontents

\section{Counting the universal case}

We now discuss the stability spaces $\tilde{P}_n$ of fine compactified universal Jacobians in genus $1$, which are discussed in \cite[Notation 6.1]{pagani2020geometry}. Note by \cite[Theorem 6.4]{pagani2020geometry} the functions $(f,g)$ with $f$ MSA are in bijection with $\overline{\mathcal{J}}^{d}_{1,n} \subseteq Simp^{d}(\overline{\mathcal{C}}_{1,n} \setminus \overline{\mathcal{M}}_{1,n})$. We wish to consider the question of \cite[Remark 6.14]{pagani2020geometry}, that is to determine the number of mildly superadditive functions $f: \mathcal{P}_{n}^{+} \rightarrow \mathbb{Z}$ with the property that $f(\{i\}) = 0$ for all $1 \le i\le n $. In particular to do so inductively, that is to determine $|\tilde{P}_n|$. As $f(\{i\}) = 0$ for all $1 \le i\le n $ it is enough to consider $f: \mathcal{P}_{n}^{+} \rightarrow \{1,\dots, n\}$.\\

Let us first introduce the language to consider such a counting problem. To find $|\tilde{P}_n|$ inductively we wish to extend an $f \in \tilde{P}_{n}$ to $\tilde{f} \in \tilde{P}_{n+1}$. This extension is equivalent to determining which $\tilde{f} \in \tilde{P}_{n+1}$ restrict to $f\in \tilde{P}_{n}$, so we aim study the fibers of the restriction map
\begin{align*} 
r:\tilde{P}_{n+1} & \rightarrow \tilde{P}_{n}\\
\tilde{f} & \mapsto \tilde{f} \mid_{\mathcal{P}_{n}^{+}}
\end{align*}

and so we have the following decomposition, 

$$\tilde{P}_{n+1} = \bigsqcup_{f \in \tilde{P}_{n}} r^{-1} (f).$$



Note that given $\tilde{f} \in \tilde{P}_{n+1}$ such that $r(\tilde{f})=f$ for $f\in \tilde{P}_n$, for $A \in \mathcal{P}_{n}^{+}$ we have $0 \le \tilde{f}(A \cup \{n+1\}) - \tilde{f}(A) \le 1$ where  $\tilde{f}(A)=f(A)$. To simplify this condition we introduce the following map $\epsilon: \mathcal{P}_{n}^{+} \rightarrow \{0,1\}$ such that 
$$\epsilon(A) = \tilde{f}(A \cup \{n+1\}) - \tilde{f}(A).$$

With this map we can rephrase $\tilde{f}$ as the map $\tilde{f}: \mathcal{P}_{n+1}^{+} \rightarrow \{1,\dots n\}$ such that for $A \in \mathcal{P}_{n}^{+}$ we have,
$$\tilde{f}(A)=f(A) \text{ and } \tilde{f}(A \cup \{n+1\}) = f(A)+\epsilon(A).$$

By construction we have a bijection between the maps $\epsilon$ and $\tilde{f}$ where $r(\tilde{f})=f$. Therefore in order to study extensions of $f$ we will study the maps $\epsilon: \mathcal{P}_{n}^{+} \rightarrow \{0,1\}$.

\begin{remark}\label{EmptyEverthing} Two special cases are where $\epsilon(S)=0$ for all $S \in \mathcal{P}^{+}_{n}$ to which we will denote by $\epsilon=0$ and similarly where $\epsilon(S)=1$ denoted by $\epsilon=1$. For each $f$ we always have $\epsilon=0$ and $\epsilon=1$ hence this shows that $|\tilde{P_{n+1}}| \ge 2|\tilde{P_n}|$.
\end{remark}

Fix a map $\epsilon: \mathcal{P}_{n}^{+} \rightarrow \{0,1\}$ and let $A,B \in \mathcal{P}_n^{+}$ with $A \subseteq B$. Additionally by fixing $\epsilon$ we have $\tilde{f} \in \tilde{P}_{n+1}$ such that $r(\tilde{f})=f$ for $f \in \tilde{P}_n$. We wish to consider the relationship between $\epsilon(A)$ and $\epsilon(B)$. As $\mathcal{P}_{n+1}^{+}$ is a poset with respect to inclusion so $A,B \in \mathcal{P}_{n+1}^{+}$ with $A \subseteq B$ in particular we have,

\begin{equation}\label{BA}
 0 \le \tilde{f}(B) - \tilde{f}(A)- \tilde{f}(B\setminus A) \le 1. 
\end{equation}

Let us give a name to this choice.

\begin{definition}\label{fminmax}
For $A,B \in \mathcal{P}_{n+1}^{+}$ we call $A \subset B$, $f$-minimal if,

$$f(B)=f(A)+f(B \setminus A)$$

or $A \subset B$ to be $f$-maximal if,

$$f(B)=f(A)+f(B \setminus A) +1.$$

\end{definition}

Take the union with the element $n+1$ and the subsets $A$ and $B$ and consider the following MSA condition for $\tilde{f}$ on $A\cup \{n+1\} \subseteq B\cup \{n+1\} $,
$$ 0 \le \tilde{f}(B \cup \{n+1\}) - \tilde{f}(A\cup \{n+1\})- \tilde{f}(B\setminus A) \le 1 .$$

Suppose $A \subseteq B$ is $f$-minimal then by (\ref{BA}) we have,

$$ 0 \le \epsilon(B) - \epsilon(A) \le 1.$$

Or if $A \subseteq B$ is $f$-maximal then similarly,

$$-1\le  \epsilon(B) - \epsilon(A) \le 0.$$

\begin{lemma}\label{epsiloncontainment}
If $A \subset B$ is $f$-minimal and if $\epsilon(A)=1$, then $\epsilon(B)=1$. Similarly if $A \subseteq B$ is $f$-maximal then if $\epsilon(B)=1$, then $\epsilon(A)=1$.
\end{lemma}

\begin{titlemize}{Questions}
\item Is $\epsilon$ the universal case for $e_{j_l}$ that is used in \cite[Lemma 3.6]{pagani2020geometry} in the description of $d_k$? 
\end{titlemize}

Now that we better understand $\epsilon$ we wish to describe a method for inductively constructing $\tilde{P}_{n+1}$ by determining all extensions $\tilde{f}$ of a particular $f \in \tilde{P}_n$. By construction of $\epsilon$ if we can characterise all possible $\epsilon$ maps from $f$, we can describe all extensions $\tilde{f}$. We can describe each $\epsilon$ by where it takes the value $1$ and satisfies lemma \ref{epsiloncontainment}. \\

We now outline the method to determine all $\epsilon$ maps for $\tilde{P}_{n+1}$. Consider the poset $\mathcal{P}_n}^{+}$ and fix an $f \in \tilde{P}_n$ by this we know for all, $A,B \in \mathcal{P}_{n}^{+}$ with $A \subseteq B$, are either $f$-minimal or $f$-maximal. For each $A$ let $\epsilon(A)=1$ and take the set elements of $\mathcal{P}_{n}^{+}$ that are upward closed with respect to inclusion in the poset for $f$-minimality and downward closed with respect to $f$-maximality. Conclude this method by taking all distinct unions of these sets. 

\begin{example}\label{graphexample}

Consider $f \in \tilde{P}_{3}$ with $f({\{1,2\})=1$, $f(\{1,2,3\})=1$ and $f(\{1,3\})=0$, which has the poset given in figure \ref{epdiagram2}. In figure \ref{epdiagram2} we colour the inclusion green if it is $f$-minimal and red if it is $f$-maximal. 

      \begin{figure}[H] 
    \centering
 \includegraphics[scale=0.3,angle=0]{Diagrams/EpsilonGraph.jpg}  
    \caption{}
    \label{epdiagram2}
\end{figure}

The set of all extensions of $f$ are given by,
\begin{align*}
E^{-1}(1) := \{& \emptyset,\\
&\{\{1,3\}\},\\
&\{\{2,3\}\},\\
&\{\{1\},\{1,3\}\},\\
&\{\{2\},\{2,3\}\},\\
&\{\{1\},\{2\},\{1,2\},\{1,3\},\{2,3\},\{1,2,3\}\},\\
&\{\{1\},\{2\},\{3\},\{1,3\},\{2,3\},\{1,2,3\}\},\\
&\{\{1\},\{2\},\{3\},\{1,2\},\{1,3\},\{2,3\},\{1,2,3\}\}\\
&\{\{1\},\{2\},\{1,3\},\{2,3\}\}\}.\\
\end{align*}

The first six non-empty extensions are derived from single nodes, the last two are from combinations of node extensions. By remark \ref{EmptyEverthing} there always exists the empty and everything extension. There are $9$ extensions of $f$ which is shown through coding calculations.

\end{example}

\begin{titlemize}{Experimental}
\item

In the code that determines $|\tilde{P}_n|$ for $n=3,4,5$ we travel  up green arrows (tail to head) and down red arrows (head to tail). By swapping the direction of the red arrows we turn the problem of finding all extensions for each $f \in \tilde{P}_n$, into one of tracing out all available walks from nodes and their taking their distinct unions.

\item For the restriction from $\tilde{P}_{4}$ to $\tilde{P}_{3}$ there are $|\tilde{P}_3|=10$ functions extending, of which each $f$ has $m \in M$ extensions to $\tilde{P}_4$ where
$$M=\{19,13\}.$$
\item Similarly for $\tilde{P}_{5}$ to $\tilde{P}_{4}$ there are $|\tilde{P}_4|=154$ functions extending, of which each $f$ has either $m \in M$ extensions for
$$M=\{99, 133, 69, 167, 42, 45, 46, 47, 111, 82, 54\}.$$
\end{titlemize}

%C:\Users\RhysL\PycharmProjects\Combinatorics2021\Database_operations\Pandas_comps_Extension_from.py


\section{Characters and group actions on $\tilde{P_n}$}

As a first coarse attempt at obtaining an understanding of the geometry of $\tilde{P}_n$, partition $\tilde{f}$ by taking sums of function values.

\begin{definition}
Let $f \in \tilde{P}_n$, define the character of $f$ to be
$$\text{Char}(f):=(\kappa_1,\dots,\kappa_n) := \{ \tilde{f} \:| \sum_{\substack {S \in \mathcal{P}_{n}^{+},\\ |S|=i}}f(S)= \kappa_i \}.$$
\end{definition}

As the function is applied to subsets of $[n]$ so there is the following natural action of the group $S_n$ on $\tilde{P}_n$.

\begin{definition}
Let $\sigma \in S_n$ and $f \in \tilde{P}_n$, we have the action

$$\sigma \cdot f(S) = f(\sigma(S)).$$
\end{definition}
 
 As $f$ takes values in $\mathbb{Z}$ so the $S_n$ orbits of $\tilde{P}_n$ are contained within characteristics of $f$ i.e $S_n \cdot f \subset Char(f)$.

\begin{titlemize}{Experimental}

\item To study the $S_n$ orbits of $\tilde{P}_n$ for $n=3,4,5$, I attached an index term to $f\in\tilde{P}_n$ to show which character orbits decomposed into $S_n$ orbits and I used this method to determine the following table.

       \begin{figure}[H] 
    \centering
 \includegraphics[scale=0.3,angle=0]{Diagrams/CharSnOrbits.jpg}  
    \caption{}
    \label{}
\end{figure}
 
\end{titlemize}

 
\begin{example}

Consider the following two MSA functions with both $f(\Delta^{3})=1$ and all faces $f(\Delta^{1})=1$ and for highlighted edges take $f(\Delta^{2})=1$ otherwise take the value $0$ on the remaining strata. Both have characteristics $(0,3,3,1)$, but have distinct $S_4$ orbits.
      \begin{figure}[H] 
    \centering
 \includegraphics[scale=0.3,angle=0]{Diagrams/DistinctSnCharOrbits.jpg}  
    \caption{}
    \label{}
\end{figure}
\end{example}



We also have the following $\mathbb{Z}_2$ action on $\tilde{P}_n$ where $1\cdot f= \bar{f}$
such, 
\begin{equation}\label{Z2funct}
    \bar{f}(A)=|A|-1-f(A).
\end{equation}

Substituting $\bar{f}$ into the MSA condition shows $\bar{f}$ is also MSA.

\begin{example}Consider the $\mathbb{Z}_2$ action on $\tilde{P}_3$. Each arrow corresponds to a $\mathbb{Z}_2$ orbit.

      \begin{figure}[H] 
    \centering
 \includegraphics[scale=0.3,angle=0]{Diagrams/Z2actionP3.jpg}  
    \caption{}
    \label{epdiagram}
\end{figure}

\end{example}


\subsection{Closed polytopes}

As discussed in (Preliminaries) to each $f \in \tilde{P}_n $ we can associate a polytope $R_f \in P_n$. By experimental calculations we have the following decomposition by dimension of $\overline{R_f}$.

\begin{example}
Let $f \in \tilde{P}_5$, in particular let $f$ be such $f(I)=1$ for all subsets containing $\{1,3,5\}$ and $\{2,4,5\}$ and $f(I)=0$ otherwise as seen in \cite[p.33]{pagani2020geometry} this has $R_f=\emptyset$. Permuting the fixed term $5 \in [5]$ by $S_5$ there are $15$ other such functions, and the $\mathbb{Z}_2$ action then gives $30$ such functions.\\

In \textit{Sage Combinatorics 2021 Notebook} we see there are $1924$ such degenerating functions (those not of top dimension) which confirms a count previously calculated by Alexander Kasprzyk. In particular we see of the $10334$ functions of $\tilde{P}_5$ the space decomposes as follows in Figure $\ref{decompP5}$.



      \begin{figure}[H] 
    \centering
 \includegraphics[scale=0.2,angle=0]{Diagrams/closepolytopes.jpg}  
    \caption{Closed polytope decomposition of $\tilde{P}_5$}
    \label{decompP5}
\end{figure}

In Sage closed polytopes with dimension $-1$ correspond to the empty set, in this case Sage tells us there are no functions such that $\overline{R}_f = \emptyset$.

\end{example}


\printindex

\bibliographystyle{alpha}
\bibliography{bibtex}


\end{document}

\documentclass[a3paper,12pt]{article}


\usepackage[utf8]{inputenc}
\usepackage[english]{babel}
\usepackage{tikz-cd}
\usepackage{amsmath,amsfonts,amssymb,amsthm}
\usepackage{mathtools}
 \usepackage{float}
\usepackage{amsthm}
\usepackage{cite}
\usepackage{datetime} % British format dates
\usepackage[cm]{fullpage}
\usepackage{url}
\usepackage{hyperref}
\usepackage{stackrel,amssymb,amsmath}
\usepackage[nottoc]{tocbibind}
\usepackage{rotating}
\usepackage[autostyle]{csquotes}
\usepackage{natbib}
\usepackage{graphicx}
%\usepackage{natbib}
\usepackage{graphicx}

\newtheorem{problem}{Problem}
\newtheorem{attempt}{Attempt}
\newtheorem{theorem}{Theorem}[section]
\newtheorem{corollary}{Corollary}[theorem]
\newtheorem{lemma}[theorem]{Lemma}
\newtheorem{proposition}[theorem]{Proposition}
\theoremstyle{definition}
\newtheorem{definition}{Definition}[section]
%\theoremstyle{indented}
\newtheorem*{remark}{Remark}
\newenvironment{titlemize}[1]{%
  \paragraph{#1}
  \begin{itemize}}
  {\end{itemize}}
  
\theoremstyle{definition}
\newtheorem{example}{Example}[section]  
\theoremstyle{definition}
\newtheorem{exercise}{Exercise}[section]  



\title{Personal Notes on Graph stabilites}
\author{Rhys Wells}
\date{\today}

\begin{document}
\maketitle
\tableofcontents



\section{Notes}

The main parts of the examples from here have been distilled into 2N_Graph_stabilities. Some questions have been left. Need further cleaning in particular the questions and to put away notes that Im not using any more. 

\begin{titlemize}{Old axiom two for $\sigma(G)$}
\item
    \item For a fixed spanning tree $\Gamma$ of $G$ there exists a set,
    
    $$\sigma^{\circ}(\Gamma)=\{\bold{d}(\Gamma) \in \mathbb{Z}^{Vert(G)} \mid \sum d(\Gamma) =-b_1(G)\}$$
    
    such that for all orientations $s:Edges(G \setminus \Gamma) \rightarrow Vert(G)$,
    
    $$\bold{d}(\Gamma) + \sum_{l \in E(G\setminus \Gamma)} \delta _{s(l)} \in \sigma(G).$$
    
    For each $\Gamma$ we denoted this set of elements as

$$\sigma(\Gamma):=\left\{\bold{d}(\Gamma) + \sum_{l \in E(G\setminus \Gamma)} \delta _{s(l)} \right\}.$$

In particular we have the following decomposition,

$$\sigma(G)= \bigcup_{\Gamma \subseteq G } \sigma(\Gamma).$$
\end{titlemize}

\subsection{Vine Curves}



\begin{example}
Vine curves with are curves have $G$ with two components and $k$ edges. Let $G_0$ be a single vertex, then  

$${\# (E(G \setminus G_0) \cap E(G \setminus G_0^{c}) )}=k.$$

As $\bold{d}=(d,-d)$ enough to consider only $d$. In particular we have the Oda-inequaliy,

\begin{equation}\label{vineOda}
|d-\phi|<\frac{k}{2}
\end{equation}

We can ask when is $\bold{d}$, $\phi$-stable? Equation (\ref{vineOda}) requires us to take $d$ to be the $k$ closest integers to $\phi$. If $k$ is even then $\phi$ is non-degenerate if and only if $\phi \notin \mathbb{Z}$. If $k$ is odd $\phi$ is non-degenerate if and only if $\phi \in \frac{1}{2}+\mathbb{Z}$.

Where spanning trees are a single edge with $|E(G \setminus \Gamma)|=k-1$, there are $2^{k-1}$ choices of orientations in proposition \ref{axiom2}, $b_1(G)=k-1$. For $d \in \mathbb{Z}$ every $\sigma(G)$ (by definition \ref{axioms12}) is of the form 

$$\sigma(G)=\left\{ (d,-d), (d+1,-d-1),\dots,(d+k-1,-d-k+1) \right\}.$$

\begin{remark}
 When $G$ has two vertices it is easy to describe $\sigma(G)$ as there is only "one direction to translate" on action of $J(G)$
\end{remark}

Take $\phi$ to be, 
$$\phi = (d+ \frac{k-1}{2}, -d -\frac{k-1}{2}).$$

By question \ref{qu1} there are no Oda inequalities. Checking axiom 2 we have 

$$\phi_{\Gamma}(v_1,v_2) = (d,-d-k+1)$$

which are integers so $d(\Gamma,\phi)=\phi_{\Gamma}$. Considering all orientations on $G \setminus \Gamma$ we have

$$d(\Gamma,\phi)(v_1,v_2) + \sum_{l \in E(G\setminus \Gamma)} \delta _{s(l)} (v_1,v_2) = (d,-d-k+1) + (a,k-1-a).$$

Only for $G_0=G$ is 

$${\# (E(G \setminus G_0) \cap E(G \setminus G_0^{c}) )}=k,$$

i.e there are no inequalities of the form (23) as $d \in S(G)$ and there are no other proper $G_0$.

The tuple given by 

$$ \frac{1}{2}\# \{l : l \in Edges(G\setminus \Gamma), v \text{ is the endpoint of } l \} (v_1,v_2) = (\frac{k-1}{2},\frac{k-1}{2}).$$

And the tuple given by 

$$\sum_{l \in E(G\setminus \Gamma)} \delta _{s(l)}(v_1,v_2)$$

for $k$ nodes sketches the following,

      \begin{figure}[H] 
    \centering
 \includegraphics[scale=0.3,angle=0]{Diagrams/delta.jpg}  
    \caption{}
    \label{}
\end{figure}


\end{example}

\subsection{Necklace curves}

\begin{exercise}
Understand and write up a proof for the case of necklace curves.
\end{exercise}

\begin{exercise}
For $\phi \in Q_G$ show $S(d,\sigma)=S_\phi $ i.e the generalised stability in the genus $1$ case mentioned in the \cite[Section 3]{pagani2020geometry}. Why does taking the average work ? Automorphisms transitive of vertices. 
\end{exercise}

Necklace curves are denoted as $I_n$ with $n$ components. The following relation is deduced from \cite[Definition 2.5]{pagani2020geometry} which reduces to definition \ref{linebundleinequal} for line bundles. In the case of necklace curves we have $1$ in the numerator as each cut for a necklace curve gives $2$ edges.  

\begin{definition}
Let $\phi \in V(X)$, the $\phi$-(semi)stable multidegrees for a necklace curve are given by the following set,
$$ S_{\phi} = \left\{ d \mid  \Bigl|{\sum_{i \in I_0} d_i - \phi_i \Bigr| \le 1,\: \forall \: \emptyset \subsetneq I_0 \subsetneq I\right\} \subseteq S(X)$$

where each $I_0$ is a consecutive set.
\end{definition}

\begin{remark}
In particular we have an $I_0$ for each component. So either $d_i = \lfloor \phi_i \rfloor$ or $d_i = \lceil \phi_i \rceil$. Additionally each necklace has $\binom{n}{2}$ cuts.
\end{remark}



\begin{exercise}
For a necklace curve $I_n$, if $\phi$ is non-degenerate show $|S_\phi|=n$.
\end{exercise}
\begin{proof}
For $\Gamma_{I_3}$ we only need to consider $|d_i-\phi_i|<1$ as $\underline{d} \in S(G)$ and $\phi \in V(G)$.
\end{proof}


For a necklace curve, let $\sigma \in S_n$ be a labelling of nodes and $\underline{d} \in S(G)$ we have,
$$S(d,\sigma) = \{\sigma^{J}(d) \mid J \in {\{1,\dots,|I|\}\} \subseteq S(X).$$

\begin{remark}
Different $d_i$ and $\sigma_i$ give the same set $S(d,\sigma)=S(d_i,\sigma_i)$. 
\end{remark}

\begin{example}
Consider $I_3$. For $\phi=\frac{1}{3},\frac{1}{3},\frac{-2}{3}$, lableing the graph anti clockwise starting from $v_1$ as the left corner.then using the oda inequalities, 

$$\sigma_{\phi_1} (G)=\{(0,0,0), (1,0,-1), (0,1,-1)\}$$

If $\phi_2=(-\frac{1}{3},-\frac{1}{3},\frac{2}{3})$,

then 
$$\sigma_{\phi_2} (G)=\{(0,0,0), (-1,0,1), (0,-1,1)\}$$


\end{example}

\begin{titlemize}{Question}
\item How is the $+1,-1$ on vertices related to the generalise stability and in particular to axiom 2? 
\end{titlemize}

\begin{example}

\end{example}

\begin{remark}
For a necklace curve we have,

$$\sum_{v \in Vert(G)} d(\Gamma,\phi)(v)+1 = \sum_{v \in Vert(G)} d(v)$$

\end{remark}

\begin{example}
Necklace curves have the following properties, $b_1(G)=1$. On \cite[p.17]{pagani2020geometry} it is noted that $S(\sigma,d)=\sigma(G)$. After labelling $Aut(G)= \mathbb{Z}/ n \mathbb{Z}$. The graph $G$ has $n$ spanning trees, in particular one up to rotation. For the Oda-inequalities it is enough to consider $I_0$ up to rotation. There is only one edge in $G \setminus \Gamma$ and so only 2 orientations.

For the tuple (up to relabelling and choice of $\Gamma$ for positioning)

$$\sum_{l \in E(G\setminus \Gamma)} \delta _{s(l)} \underline{v} = (1,0,\dots,0) \text{ or } (0,0,\dots,1) $$

and 

$$ \frac{1}{2}\# \{l : l \in Edges(G\setminus \Gamma), v \text{ is the endpoint of } l \} (\underline{v}) = (\frac{1}{2},0,\dots,0,\frac{1}{2}).$$

For the decomposition

$$\sigma_G(\phi)= \bigcup_{\Gamma \subseteq G } \sigma(\Gamma,\phi)$$

for $I_3$ the space decomposes as,

      \begin{figure}[H] 
    \centering
 \includegraphics[scale=0.3,angle=0]{Diagrams/axisdelta.jpg}  
    \caption{}
    \label{}
\end{figure}

\end{example}

\subsection{Mixed graph}
\begin{exercise} Show there exists a $\phi$ for $\sigma(G)$ of G for the simplest case of a graph where we mix the necklace and vine curve cases. (The $\phi$ case is in \cite[Example 5.4]{kass2019stability})
\end{exercise}
\begin{proof}
Need to describe $\sigma(G)$, then make a choice of $\phi$. I dont think itll be a cleaning written as in vine of necklace cases. look for counterexample.
$|\sigma_G|=5$ as there are $5$ spanning trees.
\end{proof}

\begin{example}

Consider the graph $G$,

      \begin{figure}[H] 
    \centering
 \includegraphics[scale=0.3,angle=0]{Diagrams/graphcurve.jpg}  
    \caption{}
    \label{}
\end{figure}

which has the curve,

      \begin{figure}[H] 
    \centering
 \includegraphics[scale=0.3,angle=0]{Diagrams/graphcurveCurve.jpg}  
    \caption{}
    \label{}
\end{figure}

We have $Aut(G)=\mathbb{Z}/2\mathbb{Z}$ and $J(G)=\mathbb{Z}/5\mathbb{Z}$. For $G$ we have the following $\phi$-inequalities

$$|d_1-\phi_1|<\frac{3}{2},|d_2-\phi_2|<1, |d_3-\phi_3|<\frac{3}{2}.$$

Describing $\sigma(G)$ is difficult as there are "more directions" to add $t\in J(G)$ as compared to the vine curve case.

      \begin{figure}[H] 
    \centering
 \includegraphics[scale=0.3,angle=0]{Diagrams/axiom2graphcurve.jpg}  
    \caption{}
    \label{}
\end{figure}

\begin{remark}
Understanding $\sigma(G)$ will give a better reasoning for the choice of $\phi$. How do the conditions on $\sigma(G)$ give $\phi$ that satisfy the $\phi$-inequalities?
\end{remark}

We have $\sigma(G)$ decomposing as,

$$\sigma_G(\phi)= \bigcup_{\Gamma \subseteq G } \sigma(\Gamma,\phi)$$

with overlap of $d$ for different $\Gamma$. The trees of $G$ are,

      \begin{figure}[H] 
    \centering
 \includegraphics[scale=0.3,angle=0]{Diagrams/treesofcurve.jpg}  
    \caption{}
    \label{epdiagram}
\end{figure}


\begin{titlemize}{Question}
\item Why do we need a format for $\sigma(G)$. Could we take 

$$(d_1,d_2,-d_1-d_2)$$

and then translate by $t \in J(G)$ and take the average the same as $I_5$. The issue is, these $d \in \sigma(G)$ are not near with respect to axiom 2, i.e why not take

$$(d_1+1,d_2-1,-d_1-d_2) \in S(G).$$
\item As $|\sigma(G)|=|J(G)|$ when taking the average should we be dividing by $|J(G)|$ instead of the number of vertices?



\end{titlemize}
\end{example}

\section{Material of note}

\subsection{generalised stability for a polarisation?}

\begin{definition}
Let $\phi \in Q_G$ and $\Gamma$ a spanning tree of $G$,
define 

\begin{align*}
\phi_{\Gamma} \colon Vert(G) &\to \mathbb{R} \\
v &\mapsto \phi(v) - \frac{1}{2}\# \{l : l \in Edges(G\setminus \Gamma), v \text{ is the endpoint of } l \}
\end{align*}

and we define the degree as,

\begin{align*}
d(\Gamma,\phi): Vert(G) &\rightarrow \mathbb{Z} \\
v &\mapsto \{\text{the closest integer approximation of }\phi_\Gamma(v) \}.
\end{align*}
\end{definition}

\begin{remark}
$$\sum_{v \in Vert(G)} d(\Gamma,\phi)(v) = - \#Edges(G \setminus \Gamma) = -b_1(G).$$


\end{remark}

\begin{proposition}\label{axiom2}

Let $\phi\in Q_G$, the set $\sigma_G(\phi)$ satisfies the following for all spanning trees $\Gamma$ of $G$, fixing $\Gamma$ we have for all orientations 

$$s:Edges(G \setminus \Gamma) \rightarrow Vert(G)$$
such that for $\underline{v} \in Vert(G)$,

$$d(\Gamma,\phi)(\underline{v}) + \sum_{l \in E(G\setminus \Gamma)} \delta _{s(l)}(\underline{v}) \in \sigma_G(\phi).$$

\end{proposition}


Denoted this set of elements as

$$\sigma(\Gamma,\phi):=:S_{\Gamma}(\phi) = \left\{d(\Gamma,\phi)(\underline{v}) + \sum_{l \in E(G\setminus \Gamma)} \delta _{s(l)}(\underline{v}) \right\}$$

additionally we have

$$\sigma_G(\phi)= \bigcup_{\Gamma \subseteq G } \sigma(\Gamma,\phi).$$
 
\begin{theorem}
Every fine compactified Jacobian of $X$ is isomorphic in non-natural way to $X$.
\end{theorem}

\section{Notes}

$$
S(X) = \{d : I \rightarrow \mathbb{Z} \mid \sum_{i \in I} d_i =0 \} \subseteq \mathbb{Z}^{I}$$ 

where $S(X) \subseteq V(X)$ for

$$V(X) = \{\phi : I \rightarrow \mathbb{R} \mid \sum_{i \in I} \phi_i =0\} \subseteq \mathbb{R}^{I}.$$

The space $V(X)$ contains the degenerate locus of hyperplanes and $V(X)=S(X)\otimes_{\mathbb{Z}}\mathbb{R}$. Analogously f

\subsection{Notes}

Non-example notes

Sage has built in functions to do computations on sandpile group and so $J(G)$ (which are a generalisation of undirected graphs.)

Using jacobian represenation giives the superstables in sage. 

\hline

\begin{remark}
By Dhars burning in \cite{corry2018divisors} there is a bijection between spanning trees and superstables. 
\end{remark}

\hline 

\maketitle
\tableofcontents



\printindex

\bibliographystyle{alpha}
\bibliography{bibtex}



\end{document}
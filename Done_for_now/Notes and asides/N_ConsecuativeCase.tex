\documentclass[a3paper,12pt]{article}
\usepackage[utf8]{inputenc}
\usepackage[english]{babel}
\usepackage{tikz-cd}
\usepackage{amsmath,amsfonts,amssymb,amsthm}
\usepackage{mathtools}
 \usepackage{float}
\usepackage{amsthm}
\usepackage{cite}
\usepackage{datetime} % British format dates
\usepackage[cm]{fullpage}
\usepackage{url}
\usepackage{hyperref}
\usepackage{stackrel,amssymb,amsmath}
\usepackage[nottoc]{tocbibind}
\usepackage{rotating}
\usepackage[autostyle]{csquotes}
\usepackage{natbib}
\usepackage{graphicx}
%\usepackage{natbib}
\usepackage{graphicx}

\newtheorem{problem}{Problem}
\newtheorem{attempt}{Attempt}
\newtheorem{theorem}{Theorem}[section]
\newtheorem{corollary}{Corollary}[theorem]
\newtheorem{lemma}[theorem]{Lemma}
\newtheorem{proposition}[theorem]{Proposition}
\theoremstyle{definition}
\newtheorem{definition}{Definition}[section]
%\theoremstyle{indented}
\newtheorem*{remark}{Remark}
\newenvironment{titlemize}[1]{%
  \paragraph{#1}
  \begin{itemize}}
  {\end{itemize}}
  
\theoremstyle{definition}
\newtheorem{example}{Example}[section]  
\theoremstyle{definition}
\newtheorem{exercise}{Exercise}[section]  



\title{Combinatorics of $\tilde{P}_n$}
\author{Rhys Wells}
\date{\today}

\begin{document}

\begin{proof}
To show $|Q_n=n!|$, assume $|Q_{n-1}| = n-1!$ and for each $f \in Q_{n-1}$ by the map $\psi:Q_n^{f} \rightarrow [n]$ we assign $n$ extensions $\tilde{f} \in Q_n^{f}$.

\end{proof}

\begin{remark}\label{lessconditions}
Note $Q_n$ have less MSA conditions than $\tilde{P}_n$ as $C_n \subseteq \mathcal{P}_n^{+}$. In particular when $A \subset B$ we require $B\setminus A \in C_n$.
\end{remark}

\begin{example} Can we obtain the consecutive extensions of $f$ in a similar manner to those of $\tilde{P}_n$? That is uning $\epsilon$ extensions with f-min/closed? 

\begin{titlemize}{Question}
\item Do we get $6$ extensions if we restrict from $\mathcal{P}_3^{+}$ to $C_3$ from $\tilde{P}_3^f$ in Figure \ref{epdiagram} by deleting the node $\{1,3\}$.
\end{titlemize}

Consider $f \in \tilde{P}_3$ from Example \ref{graphexample}. By deleting the node $\{1,3\}$ from Figure \ref{epdiagram} we have the following possible extensions. Restricting from $\mathcal{P}_3^{+}$ to $C_n$ by deleting the $\{1,3\}$ node and including the empty and everything set we have $8$ "extensions".
\begin{align*}
\epsilon_{f}^{-1}(1) = \{& \emptyset,\\
&\{\{1\},\\
&\{\{2\},\{2,3\}\},\\
&\{\{1\},\{2\},\{2,3\}\},\\
&\{\{1\},\{2\},\{3\},\{2,3\},\{1,2,3\}\},\\
&\{\{1\},\{2\},\{1,2\},\{2,3\},\{1,2,3\}\},\\
&\{\{1\},\{2\},\{3\},\{1,2\},\{2,3\},\{1,2,3\}\}\\
&\{\{2,3\}\},\\
\end{align*}

Consider instead $f:C_3 \rightarrow \mathbb{Z}$ such that $f(\{1,2\})=1, f(\{2,3\})=0$ and $f(\{1,2,3\})=1$.

\begin{titlemize}{Question}
\item Can $Q_3^f$ be constructed in a similar manner to $\tilde{P}_3$? Should we include the empty and everything extensions in $Q_n$? 
\item By deleting $\{1,3\}$ are these extensions the same as $Q_3^f$? Is this because of remark \ref{lessconditions}?  
\end{titlemize}

We previously agreed we would have the following for $Q_3^f$.

\begin{align*}
\epsilon_{f}^{-1}(1) = \{
&\{1\},\\
&\{\{2\},\{2,3\}\},\\
&\{\{1\},\{2\},\{2,3\}\},\\
&\{\{1\},\{2\},\{3\},\{2,3\},\{1,2,3\}\},\\
&\{\{1\},\{2\},\{1,2\},\{2,3\},\{1,2,3\}\},\\
&\{\{1\},\{2\},\{2,3\},\{1,2,3\}\}\}\\
\end{align*}

of which there are $6$.

\begin{titlemize}{Question}
\item Why is there no $\{2,3\}$, everything and $\emptyset$ extension? 
\item The notion of $f$-maximal and $f$-minimal doesn't work here for example 
$$f(123)=f(2)+f(13).$$
But there is no $\{1,3\}$, so it does not make sense for arrow at $\{2\}$ node to be red ($f$-maximal). 
\end{titlemize}

\end{example}


\maketitle
\tableofcontents



\printindex

\bibliographystyle{alpha}
\bibliography{bibtex}


\end{document}


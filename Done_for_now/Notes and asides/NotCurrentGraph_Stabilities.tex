\documentclass[a3paper,12pt]{article}
\usepackage[utf8]{inputenc}
\usepackage[english]{babel}
\usepackage{tikz-cd}
\usepackage{amsmath,amsfonts,amssymb,amsthm}
\usepackage{mathtools}
 \usepackage{float}
\usepackage{amsthm}
\usepackage{cite}
\usepackage{datetime} % British format dates
\usepackage[cm]{fullpage}
\usepackage{url}
\usepackage{hyperref}
\usepackage{stackrel,amssymb,amsmath}
\usepackage[nottoc]{tocbibind}
\usepackage{rotating}
\usepackage[autostyle]{csquotes}
\usepackage{natbib}
\usepackage{graphicx}
%\usepackage{natbib}
\usepackage{graphicx}

\newtheorem{problem}{Problem}
\newtheorem{attempt}{Attempt}
\newtheorem{theorem}{Theorem}[section]
\newtheorem{corollary}{Corollary}[theorem]
\newtheorem{lemma}[theorem]{Lemma}
\newtheorem{proposition}[theorem]{Proposition}
\theoremstyle{definition}
\newtheorem{definition}{Definition}[section]
%\theoremstyle{indented}
\newtheorem*{remark}{Remark}
\newenvironment{titlemize}[1]{%
  \paragraph{#1}
  \begin{itemize}}
  {\end{itemize}}
  
\theoremstyle{definition}
\newtheorem{example}{Example}[section]  
\theoremstyle{definition}
\newtheorem{exercise}{Exercise}[section]  



\title{Combinatorics of $\tilde{P}_n$}
\author{Rhys Wells}
\date{\today}

\begin{document}

\maketitle
\tableofcontents


\subsection{Jacobian of a graph}

For a reference see \cite[Chapter 2]{corry2018divisors}.

After choice of labelling of vertices for $G$, we have the group isomorphism $\mathbb{Z}^{Vert(G)} \cong Div(G)$ with the subgroup $S(G) \cong Div^{0}(G) \cong  \mathbb{Z}^{Vert(G)-1}$.

\begin{example}
Consider $\mathbb{Z} \xrightarrow[]{x2} \mathbb{Z}$ the image $2\mathbb{Z}$ inherits and translation on the codomain ( when J(G) acts). 
\end{example}

Elements of $\mathbb{Z}^{\text{Vert}(G)$ are firing scripts for the chip-firing action.

When applying a firing script to a chips $D$ on a graph the total degree is constant as  the number of chips fired from a vertex is equal to the sum of chips recieved from the corresponding vertices. Therefore the total degree remains constant of applying firing scripts.  

\begin{example}
Consider the banana graph with $2$ edges. The div map sends $(1,0) \mapsto (-2,2)$ and $(0,1) \mapsto (2,-2)$. For $(d,-d) \in \mathbb{Z}^{2}$ this $div((d,-d)) = (d-2,d+2)$ so if $d$ is odd it remains odd after the map, similarly if its even. 
\end{example}

\begin{definition}
Consider the linear map $div: \mathbb{Z}^{\text{Vert}(G)} \rightarrow \mathbb{Z}^{\text{Vert}(G)}$ and let $f \in \mathbb{Z}^{\text{Vert}(G)}$,


\begin{equation}
    div(f)= \sum_{v \in \text{Vert}(G)} \left( deg(v)\cdot f(v) - \sum_{vw \in E(G)} f(w) \right) \cdot v
\end{equation}


\end{definition}

The image defines the chip-firing subgroup of principal divisors
$Prin(G)= div(\mathbb{Z}^{\text{Vert}(G)}) \subseteq S(G)$.

Consider the action of $S(G)$ by the map \cite[p55]{corry2018divisors}

\begin{align*}
    S(G) \times S(G) &\rightarrow Div^{d}(G)\\
    (D,E) & \mapsto D+E.
\end{align*}

which is simply transitive. This action restricts to the action of $Prin(G)$ on $S(G)$. The Jacobian of $G$ is given as
$$J(G) = \frac{S(G)}{Prin(G)}},$$


Consider the action of $J(G)$ on $S(G)$\\

For a subset of $S\subseteq S(G)$ (in particular $\sigma(G) \subseteq S(G)$), $S$ is a complete set of representatives for the action $J(G)$ on $S(G)$ if and only if there exists a unique $s \in S$ such that $s=g(x)$ for some $g \in J(G)$ and $x \in S(G)$. \\


In axiom $1$ $J(G)$ acts on $\sigma(G)$ freely and transitively (its easy to show an action is simply transitive if and only if its free and transitive) such that for $t \in J(G)$ and $d \in \sigma(G) $, 

$$t(d):= \text{ The unique element of } \sigma(G) \text{ that represents } d+t.$$

\begin{example} For the banana graph with 2 edges axiom is
satisfied by $$\sigma(G) =\{(1,-1),(4,-4)\}$$

as it is a class of representatives, i.e it is necessary to  capture all elements of $S(G)$

but not by
$$\sigma(G) =\{(1,-1),(3,-3)\}$$ acting by $J(G)$ we only get $(d,-d)$ with $d$ odd. 

Neither satisfy axiom 2.

\begin{remark}
The set $\sigma_G(\phi)$ is a full set of representatives for the action of $J(G)$.
\end{remark}

By Kirchhoff’s Matrix-Tree Theorem \cite[Section 9.1]{corry2018divisors}, $J(G)$ is a finite abelian group whose order is the number of spanning trees in $G$ (complexity of $G$), where a spanning subtree is a subtree of $G$ that contains all vertices of $G$.\\ 



\subsection{Does there exist a $\phi$ for a generalised stability condition?}

Let $G$ be a graph and $\Gamma$ a spanning tree contained in $G$ and let $I$ be the index set for the spanning trees of $G$. We denote the set of elements $d^{'} \in \mathbb{Z}^{Vert(G)}$ such that 

$$\sum d^{'}(v) = -b_{1}(G) = \#Edges(G\setminus \Gamma)$$

by $D^{'}$. We denote the set of orientations of $G \setminus \Gamma$ to be $S=\{s:Edges(G \setminus \Gamma) \rightarrow Vert(G)$\}. Denote the set of divisors near $d^{'}$ to be,

$$\sigma^{'}(\Gamma,d^{'}
):= \{ d \in S(G) \mid d= d^{'} + \sum_{l \in E(G\setminus \Gamma)} \delta _{s(l)} \text{ for all } s \in S  \}$$


\begin{definition}\label{axioms12}
A generalised stability condition for $G$ is a subset $\sigma(G) \subseteq S(G)$ such that,

\begin{enumerate}
    \item the set $\sigma(G)$ is a complete set of representatives for the action of $J(G)$
    
\item and for each spanning tree $\Gamma_i$ for $i \in I$, there exists $d_i^{'} \in D^{'}$ such that 

$$\sigma(G) = \bigcup_{i\in I} \sigma^{'}(\Gamma_i,d_{i}^{'}).$$

\end{enumerate}
\end{definition}

\begin{remark}
Axiom 2 can be thought of as adding chips at the endpoints of $G\setminus \Gamma$ in all possible ways.
\end{remark}

\begin{remark}
By Dhars burning in \cite{corry2018divisors} there is a bijection between spanning trees and superstables. 
\end{remark}

\begin{remark}
By the free transitive $J(G)$ action we have $|\sigma(G)|=|J(G)|=|I|$, in particular $\sigma(G)$ is finite.
\end{remark}

\begin{example}

Consider the banana graph $G$ with $k$ edges and $2$ vertices. Note the spanning trees are identical. 

In the case where $k=2$. Consider $d^{'} \in D^{'}$ these have the for $(d^{'},-1-d^{'})$. And $\sigma^{'}(\Gamma,d^{'}) = \{(d^{'}+1,-1-d^{'}) , (d^{'},-d^{'}\}$.

$J(G)=\mathbb{Z}/2\mathbb{Z}$


For $k$ arbitrary we have $\bold{d}^{'} = (d^{'}, -k+1-d^{'})$. 
And $\sigma^{'}(\Gamma,d^{'}) = \{(d^{'}+k-1,-k+1-d^{'}) , \dots ,(d^{'},-d^{'})\}$

By axiom 1, $|\sigma(G)|=|J(G)|=k$ and so 

$$\sigma(G) = \{(d^{'}+k-1,-k+1-d^{'}) , \dots ,(d^{'},-d^{'})\}$$


\end{example}





\printindex

\bibliographystyle{alpha}
\bibliography{bibtex}




\end{document}

\documentclass[a3paper,12pt]{article}


\usepackage[utf8]{inputenc}
\usepackage[english]{babel}
\usepackage{tikz-cd}
\usepackage{amsmath,amsfonts,amssymb,amsthm}
\usepackage{mathtools}
 \usepackage{float}
\usepackage{amsthm}
\usepackage{cite}
\usepackage{datetime} % British format dates
\usepackage[cm]{fullpage}
\usepackage{url}
\usepackage{hyperref}
\usepackage{stackrel,amssymb,amsmath}
\usepackage[nottoc]{tocbibind}
\usepackage{rotating}
\usepackage[autostyle]{csquotes}
\usepackage{natbib}
\usepackage{graphicx}
%\usepackage{natbib}
\usepackage{graphicx}

\newtheorem{problem}{Problem}
\newtheorem{attempt}{Attempt}
\newtheorem{theorem}{Theorem}[section]
\newtheorem{corollary}{Corollary}[theorem]
\newtheorem{lemma}[theorem]{Lemma}
\newtheorem{proposition}[theorem]{Proposition}
\theoremstyle{definition}
\newtheorem{definition}{Definition}[section]
%\theoremstyle{indented}
\newtheorem*{remark}{Remark}
\newenvironment{titlemize}[1]{%
  \paragraph{#1}
  \begin{itemize}}
  {\end{itemize}}
  
\theoremstyle{definition}
\newtheorem{example}{Example}[section]  
\theoremstyle{definition}
\newtheorem{exercise}{Exercise}[section]  



\title{Combinatorics of $\tilde{P}_n$}
\author{Rhys Wells}
\date{\today}

\begin{document}

\maketitle
\tableofcontents

\section{Understanding $\tilde{P}_n$ }

 Throughout we take the hypercube case where $f(\{i\})=0$ for all $i \in [n+1]$ and $f:\mathcal{P}_n^{+} \rightarrow \mathbb{N}\cup \{0\} \in \tilde{P}_n$.

      \begin{figure}[H] 
    \centering
 \includegraphics[scale=0.2,angle=0]{Diagrams/DirectedGraph.jpg}  
    \caption{Path tracing arguement}
    \label{epdiagram}
\end{figure}
\end{titlemize}


\subsection{Induction on $\tilde{P}_n$}

We aim to describe the map $\tilde{P}_n \xrightarrow{\alpha_n}\tilde{P}_{n+1}$. In particular note for $f^{'} \in \tilde{P}_{n+1}$ we have 

\begin{equation}
\tilde{f}(A) \le \tilde{f}(A \cup \{n+1\}) \le \tilde{f}(A) +1.
\end{equation}

As preliminary notation define an pseudo-extension of $f$ to be as follows.

\begin{definition}

To pseudo-extend $f \in \tilde{P}_n$ to $f^{'}: \mathcal{P}_{n+1}^{+} \rightarrow \mathbb{N}\cup \{0\}$, define the map $\epsilon: \mathcal{P}_n^{+} \rightarrow \{0,1\}$ with $A \mapsto \epsilon(A)$ such that 

$$f^{'}(A)=f(A) \text{ and } f{'}(A \cup \{n+1\}) = f(A)+\epsilon(A).$$

\end{definition}

\begin{remark}
 Note there are many possible pseudo-extensions for a given $f$. We want those such that $f^{'}|_{\mathcal{P}_{n}^{+}} =f$ (See Example \ref{GoodMSA}) and $f^{'} \in \tilde{P}_{n+1}$. 
\end{remark}


\begin{example}\label{notMSA}
Note $f^{'}$ need not be MSA. Consider an arbitrary map $\epsilon: \mathcal{P}_n^{+} \rightarrow \{0,1\}$ "not deduced from $f$". Take $f \in \tilde{P}_3$ to be $f(\{1,2\})=1$, $f(\{1,2,3\})=1$ and $f(\{2,3\})=0$. Take $\epsilon(\{2,3\})=1$ and $\epsilon(\{3\})=1$, else take $\epsilon(I)=0$. Then $f^{'}$ will have the values

$$f^{'}(\{1,2\})=1, f^{'}(\{3,4\})=1, f^{'}(\{1,2,3\})=1, f^{'}(\{1,2,4\})=1,  f^{'}(\{2,3,4\})=1, f^{'}(\{1,2,3,4\})=1$$

and $f(I)=0$ else. By considering the inequality,

$$ 0 \le f^{'}(\{1,2,3,4\}) - f^{'}(\{1,2\}) - f^{'}(\{3,4\}) \le 1, $$

we see that $f^{'} \notin \tilde{P}_4$. 

\begin{titlemize}{Question}
\item How do you determine which $\epsilon$, allow for a valid $f^{'} \in \tilde{P}_{n+1}$? We allow $\epsilon(A)=1$ only on those $A$ that respect $f$-minimality and $f$-maximality, so that we respect the MSA conditions on all elements of $\mathcal{P}_{n}^{+}$ not  just those containing $\{n+1\}$.
\end{titlemize}

\end{example}

Following from the definition of MSA take $I \sqcup J$ to be $B$, and $A$ and $B \setminus A=A^{c}$ to be $I, J$.

\begin{definition}\label{fminmax}
For $A,B \subset [n]$ we call $A \subset B$, $f$-minimal if,

$$f(B)=f(A)+f(B \setminus A)$$

or $A \subset B$ to be $f$-maximal if,

$$f(B)=f(A)+f(B \setminus A) +1.$$

\end{definition}

\begin{remark}
By the MSA condition $f$-minimal if 
$$0 = f(B) -f(A)- f(B \setminus A)$$ 
and $f$-maximal if 
$$ f(B) -f(A)- f(B \setminus A)=1.$$

In figure \ref{epdiagram2} we colour the corresponding inclusion arrow of the poset $\mathcal{P}_{n}^{+}$ to be green and red respectively. In particular if $A \subseteq B$ is $f$-minimal then so is $A^{c} \subseteq B$, similarly for $f$-maximal. One can see this pictorially in figure \ref{epdiagram2} with respect to the green arrows, and the red arrows.
\end{remark}

\begin{example}
Let $f \in \tilde{P}_4$ with $f(\{1,3\})=1$ and $f(\{1,2,3\})=1$ otherwise $f(S)=0$. We see that $f(\{1,2,3\})=f(\{1,2\})+f(\{3\})$ and so $\{1,3\}\subseteq \{1,2,3\}$ is $f$-minimal and by symmetry so is $\{2\} \subset \{1,2,3\}$. Likewise $\{1,2\} \subset \{1,2,3\}$ is $f$-maximal as $f(\{1,2,3\}) = f(\{3\})+f(\{1,2\})+1$.
\end{example}


\begin{lemma}\label{epsiloncontainment}
Let $f\in \tilde{P}_n$ and if $A \subset B$ is $f$-minimal then if $\epsilon(A)=1$, then $\epsilon(B)=1$. Similary if $A \subseteq B$ is $f$-maximal then if $\epsilon(B)=1$, then $\epsilon(A)=1$.
\end{lemma}

\begin{proof}
As $\tilde{f}$ is MSA so,\\

(ISSUE: We dont know if $f^{'}$ is MSA yet, because if we show Exercise \ref{extension_is_MSA} then this is contradicted by example \ref{notMSA}). (Here $\tilde{f}$ means $f^{'}$.)

$$\tilde{f}(B \cup \{n+1\}) \ge \tilde{f}(A \cup \{n+1\}) +\tilde{f}(B\setminus A)$$

by definition \ref{extension} we have 

$$f(B) + \epsilon(B) \ge f(A) +\epsilon(A) +f(B \setminus A)$$

in particular as $A \subseteq B$ is $f$-minimal so 

$$\epsilon(B) \ge \epsilon(A).$$

Similarly for if $A \subset B$ is $f$-maximal.

      \begin{figure}[H] 
    \centering
 \includegraphics[scale=0.3,angle=0]{Diagrams/lemma_epsilon_fminimal.jpg}  
    \caption{}
    \label{epdiagram2}
\end{figure}

\end{proof}


Consider $f$ and the poset $ \mathcal{P}_n^{+}$ which is ordered by inclusion. For a given $\epsilon$ of $f$ denote the set of $A \in \mathcal{P}_n^{+}$ such that $\epsilon(A)=1$ to be $\epsilon^{-1}(1)$. By lemma \ref{epsiloncontainment}, the set $\epsilon^{-1}(1)$ with respect to the poset ordering is upward closed under $f$-minimality and downward closed under $f$-maximality. Denote the set of all sets $\epsilon^{-1}(1)$ as $E^{-1}(1)$.\\


%%%%%%%%%%%%%%%%%%%%%%%%%%%%%%%%%%%%%%%%%%%%%%%%55



\begin{definition}\label{extension}
(This definition may need reworking
)\\

Let $f \in \tilde{P}_n$ and define $\epsilon_f: \mathcal{P}_n^{+} \rightarrow \{0,1\}$ with $A \mapsto \epsilon_{f}(A)$ such that for $A \in \epsilon^{-1}(1) $ for $\epsilon^{-1}(1) \in E^{-1}(1)$. The map $\tilde{P}_n \xrightarrow{\alpha_n}\tilde{P}_{n+1}$ takes $f \mapsto \tilde{f}$ such that,

$$\tilde{f}(A)=f(A) \text{ and } \tilde{f}(A \cup \{n+1\}) = f(A)+\epsilon_{f}(A).$$


We call $\tilde{f}$ an extension of $f$ by $\epsilon_f$.

\end{definition}




\begin{remark}
 By definition \ref{extension}, $\tilde{f}$ are given by the $\epsilon_f$ maps and $\epsilon_f$ are determined by where they take the value $1$. We denote the set of all extensions of $f$ as 

$$Ext(f) = \{\tilde{f} \mid \alpha_n(f) = \tilde{f} \text{ for } A\in \epsilon^{-1}(1) \in E^{-1}(1)\}$$

and

$$\tilde{P}_{n+1} = \bigsqcup_{f \in \tilde{P}_n} Ext(f)$$
\end{remark}

\begin{example}\label{GoodMSA}
Consider $f \in \tilde{P}_2$ with $f(\{1,2\})=0$ and take $\epsilon(\{1\})=1$, $\epsilon(\{1,2\})=1$ and $\epsilon(\{2\})=0$. Then we have the following extension $\tilde{f} \in \tilde{P}_3$ with,
\begin{align*}
\tilde{f}(\{1,2\}) &= f(\{1,2\}) =0,\\
\tilde{f}(\{1,3\}) &= f(\{1\})+\epsilon(\{1\}) =1,\\
\tilde{f}(\{2,3\}) &= f(\{2\})+\epsilon(\{2\})=0,\\
\tilde{f}(\{1,2,3\}) &= f(\{1,2\})+\epsilon(\{1,2\})=1.\\
\end{align*}

\end{example}




\begin{exercise}\label{extension_is_MSA}
Given $\epsilon_f$ of $f \in \tilde{P}_n$ show that $\tilde{f}$ is MSA.\\

(Needs to be after defining $\epsilon^{-1}(1)$ as otherwise $\tilde{f}$ is not MSA by example \ref{notMSA}). 


\end{exercise}

\begin{proof}
We aim to show $\tilde{f}$ is MSA for all $I,J \in \mathcal{P}_{n+1}^{+}$. Focusing on connected subsets this is equivalent to the case for $A \subseteq B$ such that $A \in \mathcal{P}_n^{+}$ and $B \in \mathcal{P}_{n+1}^{+}$ where,

$$0 \le \tilde{f}(B) - \tilde{f}(A)  - \tilde{f}(B \setminus A)\le 1.$$

Focusing on the element $\{n+1\}$ this in turn reduces to the special case for $A \subseteq A \cup \{n+1\} $ with,

\begin{equation}\label{simpMSA}
0 \le \tilde{f}(A \sqcup \{n+1\}) - \tilde{f}(A)  - \tilde{f}(\{n+1\})\le 1.
\end{equation}

Hence we resolve to show (\ref{simpMSA}) (note in this case $f(A)=\tilde{f}(A)$). By definition \ref{extension} we have,

$$f(A)= \tilde{f}(A) \le \tilde{f}(A \cup \{n+1\}) \le \tilde{f}(A)+1=f(A)+1.$$ 

This gives equation (\ref{simpMSA}). 

\end{proof}

\begin{titlemize}{Question}
\item Does example \ref{notMSA} not contradict exercise \ref{extension_is_MSA}?
\end{titlemize}




\hline 


\begin{proof}
As $\tilde{f}$ is MSA, take

$$0=\tilde{f}(B \cup \{n+1\}) - \tilde{f}(A \cup \{n+1\}) -\tilde{f}(B\setminus A)$$

otherwise we have a contradiction by definition of $\epsilon$, as $\epsilon(A)=1$ and $A \subset B$ is $f$-minimal. By definition of $\tilde{f}$ we have 

$$f(B) + \epsilon(B) \ge f(A) +\epsilon(A) +f(B \setminus A)$$

in particular as $A \subseteq B$ is $f$-minimal so 

$$\epsilon(B) = \epsilon(A).$$

Similarly for if $A \subset B$ is $f$-maximal.
\end{proof}




\printindex

\bibliographystyle{alpha}
\bibliography{bibtex}


\end{document}


Please give a summary of the progress that you have made since your last report, if any. In this section you should give a brief description of the work you have completed so far, e.g. literature search completed, equipment built, experimental work still continuing etc. You should refer to the summary outcomes of your supervisory meetings and reflect on your agreed project plan and milestones for the academic session. (Minimum 300 words, maximum 750 words

\hline


I have studied the geometry of two stability spaces associated to the problem of compactifiying Jacobians in the universal case. 

Initially to understand P_n, I studied Zaslavskys hyperplane arrangements and I realised difficulties arising from visualisation in determining the n=4 case. 

Therefore I move to studying a generalisation in the form of mildly supper additive functions. To compute these I learned how to code using python and wrote code that determined all such functions for the n=3,4,5 cases, additionally the number of such functions gave the cardinality of \tilde{P_n}. Using the code I also studied the orbits of these spaces. 

I used Sage to determine the number of associated convex polytope to each functions in \tilde{P}_n. This re-derived a previous count of P_n for n=5 by Alexander Kasprzyk and suggested my code computes correctly the functions and showed there where no empty polytopes.

Showed for a single curve in genus 1 that there are n! mildly super additive functions.

I then studied the notion of \phi-stability conditions derived from the work of Oda-Seshadri.

the jacobian of a graph and the idea of a generalised stability condition for a single curve and formulated a research questions considering these ideas.

  I have gave a talk on Hermitian Symmetric Spaces in the Shimura varietiesseminar organised by Thomas Eckl.


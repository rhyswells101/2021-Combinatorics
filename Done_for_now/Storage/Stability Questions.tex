
\documentclass[a3paper,12pt]{article}
\usepackage[utf8]{inputenc}
\usepackage[english]{babel}
\usepackage{tikz-cd}
\usepackage{amsmath,amsfonts,amssymb,amsthm}
\usepackage{mathtools}
 \usepackage{float}
\usepackage{amsthm}
\usepackage{cite}
\usepackage{datetime} % British format dates
\usepackage[cm]{fullpage}
\usepackage{url}
\usepackage{hyperref}
\usepackage{stackrel,amssymb,amsmath}
\usepackage[nottoc]{tocbibind}
\usepackage{rotating}
\usepackage[autostyle]{csquotes}
\usepackage{natbib}
\usepackage{graphicx}
%\usepackage{natbib}
\usepackage{graphicx}

\newtheorem{problem}{Problem}
\newtheorem{attempt}{Attempt}
\newtheorem{theorem}{Theorem}[section]
\newtheorem{corollary}{Corollary}[theorem]
\newtheorem{lemma}[theorem]{Lemma}
\newtheorem{proposition}[theorem]{Proposition}
\theoremstyle{definition}
\newtheorem{definition}{Definition}[section]
%\theoremstyle{indented}
\newtheorem*{remark}{Remark}
\newenvironment{titlemize}[1]{%
  \paragraph{#1}
  \begin{itemize}}
  {\end{itemize}}
  
\theoremstyle{definition}
\newtheorem{example}{Example}[section]  
\theoremstyle{definition}
\newtheorem{exercise}{Exercise}[section]  



\title{Combinatorics of $\tilde{P}_n$}
\author{Rhys Wells}
\date{\today}

\begin{document}

\maketitle
\tableofcontents

\begin{exercise}
For $\phi \in Q_G$ show $S(d,\sigma)=S_\phi $ i.e the generalised stability in the genus $1$ case mentioned in the \cite[Section 3]{pagani2020geometry}. Why does taking the average work ? Automorphisms transitive of vertices. 
\end{exercise}

\begin{exercise}
For a necklace curve $I_n$, if $\phi$ is non-degenerate show $|S_\phi|=n$.
\end{exercise}

\begin{proof}
See \cite[p.17]{pagani2020geometry} for $S(\sigma,d)=\sigma(G)$.
\end{proof}

\begin{exercise} Show there exists a $\phi$ for $\sigma(G)$ of G for the simplest case of a graph where we mix the necklace and vine curve cases. (The $\phi$ case is in \cite[Example 5.4]{kass2019stability})
\end{exercise}



\end{document}